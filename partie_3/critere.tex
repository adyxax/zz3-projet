\subsection{Critères d'évaluation}

Nous avons établit la liste des critères d'évaluation à partir du cahier des charges que nous nous sommes fixé en première partie de ce rapport. Ce cahier des charges comprenait les 5 axes suivants :
\begin{itemize}
 	\item Facilité de déploiement du serveur.
	\item Client VPN multiplate-forme.
	\item Facilité d'installation, de configuration de d'utilisation du client.
	\item Intégration au sein du réseau de l'ISIMA.
	\item Niveau de sécurité fourni par la solution.
\end{itemize}
~

De ce cahier des charges on dégage rapidement trois catégories de critères à étudier : l'aspect client, le côté serveur, ainsi que le niveau de sécurité.

\subsubsection{Côte Client}

\paragraph{Déploiement du client}
~

Lors de la mise en place de chacune des solutions, nous avons présenté le processus d'installation et de mise en place du client pour l'utilisateur. Nous nous limiterons à un comparatif des clients sous Windows car c'est le seul système pour lequel les clients de chaque solution sont fonctionnels.

~

Pour la solution VPN de Microsoft, le client étant intégré au système d'exploitation, l'utilisateur a juste à créer une nouvelle connexion et la paramètrer en mode VPN qui prend plus ou moins 30 secondes en suivant les directives présentées en \ref{label_client_windows} page \pageref{label_client_windows}.

Concernant OpenVPN, l'utilisateur doit dans un premier temps télécharger le client VPN depuis le site officiel (qui aurait éventuellement sa place sur l'Intranet de l'ISIMA). Le client est très léger (978 Ko), mais la durée nécessaire à l'installationdépend des capacité de la machine, le client devat installer une carte réseau virtuelle pour son bon fonctionnement. Néanmoins, une fois installé il suffit de copier le fichier de configuration ainsi que le certificat et la clef utilisateur dans le dossier d'installation. L'ensemble prend 5 à 10 minutes.

La solution CISCO impose, de même que pour OpenVPN, de télécharger leur client. Celui-ci est plus conséquent (9 Mo) et aurait également sa place sur l'intranet de l'ISIMA. L'installation est par contre rapide (moins de 2 minutes). Pour la configuration du client, l'utilisateur a le choix. Soit il configure manuellement la connexion VPN, cela sous-entendant qu'il en connaisse les paramètres, soit il importe directement un fichier de configuration (environ 1Ko). La procédure prend en tout de 5 à 10 minutes.

~

D'un point de vue déploiement du client VPN, la solution de Microsoft reste donc la plus avantageuse sous Windows.

\paragraph{Délai de connexion}
~\

Nous nous sommes intéressés à évaluer la durée nécessaire à l'établissement de la connexion VPN. Avec le client CISCO, le délai de connexion est assez long. En effet, la connexion s'effectue en deux phases, avec tout d'abord l'authentification IKE, et ensuite l'identification en interrogeant le serveur RADIUS qui lui même interroge la base NIS.

Avec le client Windows, nous avons remarqué que le temps de connexion était plus rapide qu'avec le client CISCO. Cependant, la phase de challenge MD5 peut être longue en fonction de la charge réseau à laquelle le serveur est soumis.

Avec le client OpenVPN, l'utilisateur s'authentifie et s'identifie très rapidement en comparaison des deux autres solutions. Nous pensons que la présence d'un certificat signé sur le poste utilisateur accélère cette phase de connexion.

~

Concernant le délai de connexion, le client de OpenVPN s'avère être le plus performant. Il est à noter que nous n'avons pas souhaité mésurer avec précision le délai de connexion, car ne s'agissant que d'une maquette sur un réseau local une telle mesure n'aurait pas été représentative.

\subsubsection{Côté Serveur}

\paragraph{Déploiement du serveur}
~

L'installation du serveur OpenVPN a été particulièrement longue. En effet, il a fallu dans un premier temps installer sur le serveur CentOS l'ensemble des dépendances nécessairesà la compilation d'OpenVPN, à le compiler, puis enfin à installer le logiciel. Une fois l'installation terminée, il nous a fallu générer les différents certificats pour les groupes PROFS et STUDENTS. En dernier temps il a fallu interfacer le serveur avec la base NIS de l'ISIMA pour récupérer les identifiants et mots de passes des utilisateurs.

Pour Windows Server, nous avons dû installer quatre services différents : DNS, DHCP, Active Directory, Service d'accès distant. Une fois cette phase achevée, la configuration de chacun de ses services a été effectuée rapidement en comparaison d'OpenVPN.

Enfin pour la solution CISCO, le serveur VPN est inclus dans le matériel. Toutefois nous avons dû installer un serveur RADIUS sur le serveur Linux afin que celui-ci consulte la base NIS pour authentifier l'utilisateur. L'avantage de la solution CISCO est que le fichier de configuration du routeur peut-être sauvegardé et réutilisé. Si un jour le routeur tombe en panne, il faudrait moins de dix minutes pour que le nouveau routeur démarre et récupère la configurationvia TFTP.

~

On se rend compte rapidement que la solution CISCO est la plus adapté du point de vue temps d'installation mais aussi du point de vue de l'administration requise.

Rappelons qu'aucune des trois solutions n'est indépendante et autonome. En effet, chacune d'entre-elles a besoin d'un autre serveur pour fonctionner : Windows a besoin de se synchroniser avec le serveur ISIMA pour répliquer son Active Directory, OpenVPN fonctionne conjointement avec la base NIS, quant au routeur il dépend du serveur RADIUS qui dépend du NIS.

\paragraph{Coût de chaque solution}
~

Il nous était impossible de proposer la meilleure solution VPN sans évoquer l'aspect financier. En effet chacune des ces solutions a un coût, à commencer par la solution CISCO qui est la plus onéreuse des trois, à cause de l'investissement matériel nécessaire. Pour Microsoft, la coût de la solution est à imputer à la licence, tandis que OpenVPN étant Open-Source le coût est nul.

~

D'un point de vue coût on se rend compte que la solution OpenVPN est la plus avantageuse par rapport au deux autres.

\subsubsection{Sécurité}

\paragraph{Authentification et identification}
~

Ce critère concerne la manière dont un client est authentifié et identifié. Pour la solution Microsoft, Le système d'authentification en place est un système faible à base de clefs partagées. L'identification passe par un annuaire Active Directory local référençant les identifiants et mots de passes. Rappelons que l'une des faiblesses de l'identification avec MS-Chap v2 était que l'identifiant utilisateur ciscule en clair sur le réseau.

La solution CISCO utilise également une authentification via clefs partagées. Par contre l'identification faitintervenir un serveur RADIUS qui lui même consulte la base NIS de l'école afin de récupérer les identifiants de l'ISIMA.

Quant à OpenVPN l'authentification se fait par l'intermédiaire de certificats signés. Il s'agit du moyen le plus sûr mis en place sur la plate-forme. L'identification passe par une consultation de la base NIS.

~

D'un point de vue authentification et identification, la solution OpenVPN est la plus fiable. Rappelons malgré tout que la solution CISCO peut également fonctionner avec des certificats mais nous ne sommes jamais parvenus à mettre ce système en place, d'où notre choix pour OpenVPN.

\paragraph{Efficacité des algorithmes de chiffrement}
~

Lors de l'implémentation des trois solutions, nous avons remarqué une différence sur le type de chiffrement exploités par les solutions. Microsoft utilise un chiffrement symétrique de type MPPE avec une clef de 128 bits. Un inconvénient est que les systèmes d'exploitation antérieur à Windows XP sont incapables de le gérer, et réclameraient un chiffrement plus faible.

La solution CISCO utilise elle aussi un chiffrement symétrique mais sur 168 bits. OpenVPN quant à lui utilise une première phase de négociation en chiffrement asymétrique basé sur des certificats et clefs RSA de longueur 1024bits, puis se rabat sur un chiffrement symétrique 256 bits.

~

La comparaison au niveau sécurité élimine clairement la solution de Microsoft (dans l'état où nous l'avons configurée) de la liste des solutions envisageable. Nous ne pouvons départager OpenVPN et CISCO car les deux types de chiffrements se vaudraient si seulement l'authentification via certificats avait fonctionné sur le routeur.
