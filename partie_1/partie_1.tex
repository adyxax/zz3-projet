\section{Introduction à l'étude}
\subsection{Sujet de l'étude}
L'objet de cette étude est d'évaluer différentes technologies permettant de mettre en place des connexions sécurisées via des Réseaux Privés Virtuels (VPN). L'objectif est de recenser plusieurs solutions fonctionnant sur divers systèmes d'exploitation et de les confronter entre-elles. Les différentes solutions seront d'abord évaluées en termes de complexité d'installation pour l'utilisateur final, puis en termes de performances, de confort d'utilisation, etc. Ce travail est effectué en vue de mettre en place une solution de VPN au sein de l'ISIMA.

% \subsection{Contexte de travail}
\subsection{Architecture étudiée}
\subsubsection{Schéma logique}
\subsubsection{Schéma physique}


VPN, ou Réseau Privé Virtuel, est le nom donné à une technologie permettant de relier des machines de façon sécurisée à travers un réseau non sécurisé comme internet. Il existe deux familles de technologies VPN : les VPN basés sur IPsec qui sont la technologie la plus ancienne, et les VPN basés sur SSL qui sont eux plus récents. (TODO : et pptp? et pourquoi)

% \subsection{Architecture étudiée et objectifs fixés}
Le travail a été effectué intégralement dans la salle A214 de l'école. Nous avons pu mettre en place une maquette sur laquelle travailler, simulant les accès depuis Internet vers le réseau interne de l'ISIMA. La figure ref{schema-logique-maquette} présente un schéma logique de la maquette.

\subsection{Etat de l'art}
\subsubsection{Les différents types de VPN}
\subsubsection{Les classes de protocoles}
\paragraph{IPsec}
~

IPsec est une suite protocolaire de niveau 3, visant à apporter la sécurité manquant au protocole IP. Cette suite utilise plusieurs protocoles au cours des différentes phases de mise en place d'IPsec.

La première phase est une phase négociation au cours de laquelle les parties se mettent d'accord sur les algorithmes de chiffrement à utiliser et échangent des clefs de session via le protocole ISAKMP (Internet Security Association and Key Management Protocol). Au cours de cette phase le protocole IKE (Internet Key Exchange) intervient également pour générer ces clefs de session soit grâce à une clef partagée soit à l'aide de certificats RSA.

La seconde phase est la phase de communication au cours de laquelle les données peuvent traverser le tunnel et sont chiffrées. Deux protocoles peuvent intervenir en fonction de la finalité du tunnel : ESP qui fournit à la fois intégrité et confidentialité des données, et AH qui fournit l'intégrité et l'authentification.


% \paragraph{VPN basés sur pptp}
% \paragraph{VPN basés sur TLS}


\subsection{Objectifs fixés}
Recenser les technologies de VPN, mettre en place une maquette de test, confronter les différentes solutions.

\pagebreak
