\section{Introduction à l'étude}
\subsection{Sujet de l'étude}
L'objet de cette étude est d'évaluer différentes technologies permettant de mettre en place des connexions sécurisées via des Réseaux Privés Virtuels (VPN). L'objectif est de recenser plusieurs solutions fonctionnant sur divers systèmes d'exploitation et de les confronter entre-elles. Les différentes solutions seront d'abord évaluées en termes de complexité d'utilisation pour l'utilisateur final, puis en termes de performances pures, le travail étant effectué en vue de mettre en place une solution de VPN au sein de l'ISIMA.

\subsection{Contexte de travail}
\subsubsection{VPN : définition}
VPN, ou Réseau Privé Virtuel, est le nom donné à une technologie permettant de relier des machines de façon sécurisée à travers un réseau non sécurisé

 de créer des tunnels sécurisés (SSL/TLS) à travers un réseau non sûr comme Internet. OpenVPN est à la fois facile à installer et à configurer, en plus d'être disponible sur à peut prêt toutes les plates-formes (Linux, Windows, BSD, Mac OS, Solaris). Le principe de configuration reste le même quelque soit la plate-forme utilisée.

OpenVPN est basé sur une architecture client-server. Le VPN fonctionne soit site-à-site, soit avec des clefs partagées. Les données sont tunnelisées sur un seul port, TCP ou UDP.


\subsubsection{Les différents types de VPN}

Le travail a été effectué intégralement dans la salle A214 de l'école. Nous avons pu mettre en place une maquette sur laquelle travailler, simulant les accès depuis Internet vers le réseau interne de l'ISIMA. La figure \ref{schéma-logique-maquette} présente la façon dont  fonctionnement

\subsection{Objectifs fixés}
Recenser les technologies de VPN, mettre en place une maquette de test, confronter les différentes solutions.

\pagebreak
