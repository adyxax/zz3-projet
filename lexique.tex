\section*{Lexique}
\addcontentsline{toc}{section}{Lexique}

\begin{description}
	\item [CSV:] Coma Separated Value, ou valeurs séparées par des virgules. Format de fichier de données.
	\item [GNUPlot:] Boîte à outils permettant, entre autres, d'automatiser la génération de graphiques.
	\item [Open-Source:] Se dit d'un logiciel dont la licence correspond à certains critères comme le libre accès à son code source ainsi que sa libre redistribution.
	\item [OpenSER:] Proxy SIP reconnu pour ses excellentes performances.
	\item [Pipe nommé:] Un pseudo fichier permettant la redirection d'une sortie sur une entrée.
	\item [Protocole de transport:] Protocole dont le rôle consiste à délivrer les données aux applications.
	\item [Proxy:] Serveur jouant un rôle dans la sécurité des réseaux, servant intermédiaire aux transactions.
	\item [Raptor:] Outil de génération de trafic RTP interne à FT R\&D.
	\item [RTP:] Protocole applicatif d'échange de données audio et vidéo.
	\item [Script:] Programme s'exécutant sans compilation.
	\item [Table ARP:] Table de correspondance entre les adresses physiques (MAC) et logiques (IP).
	\item [TCP:] Protocole de transport fiable au dessus de IP.
	\item [TLS:] Protocole de session sécurisé et reposant sur TCP.
	\item [SDP:] Protocole de description de session sur lequel s'appuie SIP.
	\item [SIP:] Protocole applicatif permettant de gérer des sessions multimédia, et notamment de téléphonie.
	\item [UDP:] Protocole de transport de type \textit{best-effort} au dessus de IP.
	\item [Socket:] Interface logicielle permettant l'utilisation des ressources réseau sur une machine.
	\item [VoIP:] Qualificatif de l'ensemble des technologies vouées au transport de la voix sur un réseau IP.
	\item [XML:] Langage permettant le stockage de données quelconques sous une forme standardisée.
\end{description}

\pagebreak