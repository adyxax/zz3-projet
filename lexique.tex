\section*{Lexique}
\addcontentsline{toc}{section}{Lexique}

\begin{description}
	\item [Annuaire:] Base de données contenant les informations des utilisateurs.
	\item [Chiffrement:] Procédé cryptographique permettant de rendre illisible un document.
	\item [Open-Source:] Se dit d'un logiciel dont la licence correspond à certains critères comme le libre accès à son code source ainsi que sa libre redistribution.
	\item [NIS:] Network Identification System. Protocole standart pour l'échange d'identifiants sous Unix.
	\item [PAM:] Pluggable Authentification Modules, Système d'authentification standart sous Unix.
	\item [Protocole de transport:] Protocole dont le rôle consiste à délivrer les données aux applications.
	\item [Proxy:] Serveur jouant un rôle dans la sécurité des réseaux, servant d'intermédiaire aux transactions.
	\item [TCP:] Protocole de transport fiable au dessus de IP.
	\item [TLS:] Protocole de session sécurisé et reposant sur TCP.
	\item [UDP:] Protocole de transport de type \textit{best-effort} au dessus de IP.
	\item [Socket:] Interface logicielle permettant l'utilisation des ressources réseau sur une machine.
\end{description}

\pagebreak
