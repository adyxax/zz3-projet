\begin{abstract}

~

\textbf{VPN} est le nom donné à une technologie permettant de mettre en place une connexion sécurisée avec un réseau interne comme celui de l'ISIMA. En l'absence d'une telle technologie d'accès, les élèves et professeurs de l'ISIMA sont réduits à utiliser des moyens mal adaptés et contraignants pour accéder aux machines de l'école.

~

Ce projet a été conduit en vue d'étudier des moyens technologiques pour palier à ce manque. Il existe un grand nombre de solutions parmi lesquelles trois ont été considérées comme représentative du marché. Deux d'entre-elles sont propriétaires, l'une dépendant des technologies \textbf{CISCO}, l'autre fonctionnant dans un environnement \textbf{Windows}. La dernière est issue du monde \textbf{Open-Source} et s'exécute sous Linux.

~

L'étude repose sur la mise en place d'une plateforme de test mettant bien en évidence la problématique de la \textbf{sécurité} au sein des \textbf{VPN}. Le niveau d'intégration des différentes solutions au coeur du réseau existant est un critère déterminant du projet.

~

% L'un des critères de choix est son \textbf{niveau de sécurité}. En effet, le but d’un VPN est de pouvoir faire transiter des paquets sécurisés afin que personne ne puisse les déchiffrer en cas d’interception.
% Nous avons pu établir plusieurs critères de choix afin d’évaluer la meilleure solution comme par exemple le déploiement d'un serveur VPN, ou la compatibilité du client VPN sur différents systèmes d'exploitation.



\vfill

Mots-Clés: VPN, Open-Source, CISCO, Windows, sécurité.

\end{abstract}

\pagebreak

\renewcommand{\abstractname}{Abstract}
\begin{abstract}

A \textbf{VPN} is a technology that allow a user to have a secure connection with a local area network like ISIMA. Without this kind of technology, students and professors of ISIMA are forced to used means not user-friendly to access to ISIMA's machines.

~

This project has been made to study different kind of technology to solve this problem. There are a lot of solutions, but only three represents the market. Two of them are proprietary, the first using \textbf{CISCO}'s technology, the other working in a Windows environment. The last one is a \textbf{Open-Source} solution working with Linux.

~

The study is based on an implementation of a test bed showing the problematic of the \textbf{security} in the \textbf{VPN}. The integration level of those solutions in a existing network core is a decisive critera of the project.

\vfill

Keywords: VPN, Open-Source, CISCO, Windows, security

\end{abstract}

\pagebreak
