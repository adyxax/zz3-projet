\begin{abstract}

Un \textbf{VPN} est une connexion sécurisée qui permet à une personne extérieur d'avoir accès au réseau local de l'ISIMA. Actuellement, les élèves de l'ISIMA utilisent le protocole SSH afin de pouvoir récupérer les travaux stockés sur le réseau interne.


Le but du projet, est de faire une étude sur la mise en place d'une solution VPN, en utilisant \textbf{Windows Server 2003}, un logiciel libre sous linux \textbf{OpenVPN} et une solution matérielle avec un \textbf{routeur CISCO}.
Nous avons pu grâce à ses trois solutions tester les différents types de \textbf{chiffrement} que ces solutions proposent.


L'un des critères de choix est son \textbf{niveau de sécurité}. En effet, le but d’un VPN est de pouvoir faire transiter des paquets sécurisés afin que personne ne puisse les déchiffrer en cas d’interception.
Nous avons pu établir plusieurs critères de choix afin d’évaluer la meilleure solution comme par exemple le déploiement d'un serveur VPN, ou la compatibilité du client VPN sur différents systèmes d'exploitation.



\vfill

Mots-Clés: VPN, OpenVPN, routeur CISCO, Windows Server 2003, chiffrement, niveau de sécurité

\end{abstract}

\pagebreak

\renewcommand{\abstractname}{Abstract}
\begin{abstract}

bla bla bla

\vfill

Keywords: .

\end{abstract}

\pagebreak
