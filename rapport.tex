\documentclass[12pt, a4paper, titlepage, oneside]{article}

%%%%%% On commence par enjoliver %%%%%%%%%%%%%%%%%%%%%%%%%%%%%%%%%%%%%%%%%%%%%
\usepackage[francais]{babel}	%typographie française
\usepackage[utf8]{inputenc}	% accents 8 bits dans le source
\usepackage[T1]{fontenc}	% accents dans le DVI
\usepackage{lmodern}	% Sans cette ligne la typo va être merdique!!!

\pdfcompresslevel=9
\usepackage[pdftex]{graphicx}
\DeclareGraphicsExtensions{.jpg, .png , .pdf}


%%%%%% on inclue de l'utilitaire %%%%%%%%%%%%%%%%%%%%%%%%%%%%%%%%%%%%%%%%%%%%%
\usepackage{float}
\usepackage{graphicx}
\newcommand{\Hrule}{\rule{\linewidth}{0.5mm}}
\usepackage{listings}
% \usepackage[colorlinks=true,urlcolor=blue,linkcolor=blue]{hyperref}

%%%%%% on configure des marges mouzat-compliant (c'est pas terrible quand on utilise des foot-notes, à améliorer)
\usepackage{vmargin}
\setmarginsrb{1.5cm}{1.5cm}{1.5cm}{1.5cm}{15.71402pt}{1.5cm}{0.5cm}{2cm}

%%%%%% On personnalise la numérotation qui par défaut s'arrête aux subsubsections
\setcounter{secnumdepth}{4}	% active la numerotation auto jusqu'au niveau des paragraphes
\renewcommand{\theparagraph}{\alph{paragraph})}	% et avec une lettre plutôt qu'une plaque d'immatriculation

%%%%%% puis de jolis headers/footers %%%%%%%%%%%%%%%%%%%%%%%%%%%%%%%%%%%%%%%%%
\usepackage{fancyhdr}
\pagestyle{fancy}
\renewcommand{\sectionmark}[1]{\markright{\thesection\ #1}}
\renewcommand{\headrulewidth}{0.5pt}	% un trait en haut des pages
\renewcommand{\footrulewidth}{0.5pt}	% et un autre en bas
\fancyhead{}	% commande pour effacer le header
\fancyfoot{}	% commande pour effacer le footer


\begin{document}
%%%%%% page de garde %%%%%%%%%%%%%%%%%%%%%%%%%%%%%%%%%%%%%%%%%%%%%%%%%%%%%%%%%
	\begin{titlepage}
	\begin{minipage}{0.5\textwidth}
		\begin{flushleft} \large
			\includegraphics[height=1cm]{images/logo-isima.png}\\
			~\\
% 			Campus de Clermont Fd\\
% 			Les Cézeaux BP 10125\\
% 			63173 AUBIERE
			Institut Supérieur\\
			d'Informatique, de\\
			Modélisation et de\\
			leurs Applications
		\end{flushleft}
	\end{minipage}
	\begin{minipage}{0.43\textwidth}
		\begin{flushright} \large
 			\includegraphics[height=4cm]{images/logo.png}\\%-orange.png}\\
 			~\\
		\end{flushright}
	\end{minipage}

	\vfill
	\begin{center}
		\Hrule \\[0.4cm]
		\Large{Rapport de projet de dernière année d'école d'ingénieur}\\
		\Large Filière 5 : Infrastructure entreprise, réseaux et télécoms\\[1.2cm]
		\Huge{Etude d'une architecture VPN pour l'ISIMA}\\[0.4cm]
% 		\Huge{Etat de l'art}\\[1.5cm]
% 		\textsc{\Large(tome i)} \\[0.2cm]
		\Hrule \\[0.4cm]
% 		\Large{Durée : Cinq mois du 7 Avril au 5 Septembre}
	\end{center}
	
	\vfill
	\begin{minipage}{0.5\textwidth}
		\begin{flushleft} \large
			\emph{Auteurs:}\\
			Léonardo \textsc{Coscia}\\
			Julien \textsc{Dessaux}
		\end{flushleft}
	\end{minipage}
	\begin{minipage}{0.45\textwidth}
		\begin{flushright} \large
		\emph{Responsable ISIMA:}\\
			Christophe \textsc{Gouinaud}%\\
% 			\emph{Responsable entreprise:} \\
% 			Pascal \textsc{Queyroux}
		\end{flushright}
	\end{minipage}
	
	\vfil
	\begin{center}
		{\large 2008-2009}
	\end{center}
\end{titlepage}

% 	\begin{titlepage}
	\begin{minipage}{0.5\textwidth}
		\begin{flushleft} \large
			\includegraphics[height=1cm]{images/logo-isima.png}\\
			~\\
% 			Campus de Clermont Fd\\
% 			Les Cézeaux BP 10125\\
% 			63173 AUBIERE
			Institut Supérieur\\
			d'Informatique, de\\
			Modélisation et de\\
			leurs Applications
		\end{flushleft}
	\end{minipage}
	\begin{minipage}{0.43\textwidth}
		\begin{flushright} \large
 			\includegraphics[height=4cm]{images/logo.png}\\%-orange.png}\\
 			~\\
		\end{flushright}
	\end{minipage}

	\vfill
	\begin{center}
		\Hrule \\[0.4cm]
		\Large{Rapport de projet de dernière année d'école d'ingénieur}\\
		\Large Filière 5 : Infrastructure entreprise, réseaux et télécoms\\[1.2cm]
		\Huge{Etude d'une architecture VPN pour l'ISIMA}\\[0.4cm]
% 		\Huge{Etat de l'art}\\[1.5cm]
% 		\textsc{\Large(tome i)} \\[0.2cm]
		\Hrule \\[0.4cm]
% 		\Large{Durée : Cinq mois du 7 Avril au 5 Septembre}
	\end{center}
	
	\vfill
	\begin{minipage}{0.5\textwidth}
		\begin{flushleft} \large
			\emph{Auteurs:}\\
			Léonardo \textsc{Coscia}\\
			Julien \textsc{Dessaux}
		\end{flushleft}
	\end{minipage}
	\begin{minipage}{0.45\textwidth}
		\begin{flushright} \large
		\emph{Responsable ISIMA:}\\
			Christophe \textsc{Gouinaud}%\\
% 			\emph{Responsable entreprise:} \\
% 			Pascal \textsc{Queyroux}
		\end{flushright}
	\end{minipage}
	
	\vfil
	\begin{center}
		{\large 2008-2009}
	\end{center}
\end{titlepage}

%%%%%% résumé %%%%%%%%%%%%%%%%%%%%%%%%%%%%%%%%%%%%%%%%%%%%%%%%%%%%%%%%%%%%%%%%
	\begin{abstract}

Un \textbf{VPN} est une connexion sécurisée qui permet à une personne extérieur d'avoir accès au réseau local de l'ISIMA. Actuellement, les élèves de l'ISIMA utilisent le protocole SSH afin de pouvoir récupérer les travaux stockés sur le réseau interne.


Le but du projet, est de faire une étude sur la mise en place d'une solution VPN, en utilisant \textbf{Windows Server 2003}, un logiciel libre sous linux \textbf{OpenVPN} et une solution matérielle avec un \textbf{routeur CISCO}.
Nous avons pu grâce à ses trois solutions tester les différents types de \textbf{chiffrement} que ces solutions proposent.


L'un des critères de choix est son \textbf{niveau de sécurité}. En effet, le but d’un VPN est de pouvoir faire transiter des paquets sécurisés afin que personne ne puisse les déchiffrer en cas d’interception.
Nous avons pu établir plusieurs critères de choix afin d’évaluer la meilleure solution comme par exemple le déploiement d'un serveur VPN, ou la compatibilité du client VPN sur différents systèmes d'exploitation.



\vfill

Mots-Clés: VPN, OpenVPN, routeur CISCO, Windows Server 2003, chiffrement, niveau de sécurité

\end{abstract}

\pagebreak

\renewcommand{\abstractname}{Abstract}
\begin{abstract}

bla bla bla

\vfill

Keywords: .

\end{abstract}

\pagebreak

%%%%%% remerciements %%%%%%%%%%%%%%%%%%%%%%%%%%%%%%%%%%%%%%%%%%%%%%%%%%%%%%%%%
	\fancyhead[L]{Mise en place d'un VPN à l'ISIMA}
	\fancyhead[R]{Remerciements}
	\fancyfoot[L]{F5}
	\fancyfoot[C]{}
	\fancyfoot[R]{Rapport de projet 2009}
	\section*{Remerciements}

Nous tenons à remercier ... :

\begin{itemize}
	\item Christophe Gouinaud pour ...
	\item Patrice Laurençot pour ...
	\item ?
\end{itemize}

\pagebreak

%%%%%% table des matières %%%%%%%%%%%%%%%%%%%%%%%%%%%%%%%%%%%%%%%%%%%%%%%%%%%%
	\fancyhead[R]{Table des matières}
	\fancyfoot[C]{}
	\tableofcontents
	\pagebreak
%%%%%% on rétablie la numérotation à partir d'ici %%%%%%%%%%%%%%%%%%%%%%%%%%%%
	\fancyfoot[C]{\thepage}
	\pagenumbering{arabic}
%%%%%% Introduction %%%%%%%%%%%%%%%%%%%%%%%%%%%%%%%%%%%%%%%%%%%%%%%%%%%%%%%%%%
	\fancyhead[R]{Introduction}
	\section*{Introduction}
\addcontentsline{toc}{section}{Introduction}



\pagebreak

%%%%%% Présentation de l'entreprise %%%%%%%%%%%%%%%%%%%%%%%%%%%%%%%%%%%%%%%%%%
% 	\fancyhead[R]{Présentation de l'entreprise}
% 	\input{presentation_entreprise.tex}
%%%%%% Et enfin l'étude %%%%%%%%%%%%%%%%%%%%%%%%%%%%%%%%%%%%%%%%%%%%%%%%%%%%%%
	\fancyhead[R]{\rightmark}
	\section{Introduction à l'étude}
\subsection{Sujet de l'étude}
L'objet de cette étude est d'évaluer différentes technologies permettant de mettre en place des connexions sécurisées via des Réseaux Privés Virtuels (VPN). L'objectif est de recenser plusieurs solutions fonctionnant sur divers systèmes d'exploitation et de les confronter entre-elles. Les différentes solutions seront d'abord évaluées en termes de complexité d'installation et d'administration, puis en termes de performances et de facilité d'utilisation.

VPN, ou Réseau Privé Virtuel, est le nom donné à une technologie permettant de relier des machines de façon sécurisée à travers un réseau non sûr comme Internet. Ce travail est effectué en vue de mettre en place une telle solution d'accès au sein de l'ISIMA. Les avantages d'une telle installation seraient multiples, du fait de la possibilité d'accéder au réseau interne de l'école depuis l'extérieur dans les mêmes conditions que si l'on était à l'intérieur.

Deux catégories de solutions VPN existent. On distingue les plate-formes dédiées tels les concentrateurs que les grands équipementiers proposent, et les solutions logicielles disponibles sur les systèmes d'exploitation courants : Microsoft a sa solution, le monde de l'Open-Source également.

Trois architecture ont donc été étudiées afin d'être en accord avec la diversité des solutions existantes : un routeur CISCO 2811XM configuré pour faire du VPN, un WindowsServer2K3 avec les services requis pour établir de connexions VPN, ainsi qu'un serveur Linux CentOS 5.1 avec le logiciel libre OpenVPN. La mise en place d'une plate-forme de test sera au coeur de cette étude.

% \subsection{Contexte de travail}
\subsection{Architecture étudiée}
\label{section_architecture_etudiee}
\subsubsection{Schéma logique}

Le travail a été intégralement réalisé dans la salle A214 de l'école. Nous avons pu mettre en place une maquette complète, permettant de simuler les accès depuis Internet vers un réseau interne. La figure \ref{schema-logique-maquette} présente un schéma logique de la maquette.

\begin{figure}[H]
	\begin{center}
 		\includegraphics[width=\textwidth]{partie_1/images/archi_logique.png}\\
	\end{center}
	\caption{Schéma logique de la maquette}
	\label{schema-logique-maquette}
\end{figure}

\subsubsection{Schéma physique}

Pour faire cohabiter physiquement les trois architectures ensembles nous avons dû faire des conscessions. La plus importante a consisté à récupérer en DHCP les adresses IP des interfaces connectées au réseau de l'ISIMA, dans un coucis d'interropérabilité. La figure \ref{schema-physique-maquette} illustre l'architecture mise en place :

\begin{figure}[H]
	\begin{center}
	\includegraphics[width=\textwidth]{partie_1/images/archi_phy.png}\\
	\end{center}
	\caption{Schéma physique de la maquette}
	\label{schema-physique-maquette}
\end{figure}

\subsection{Etat de l'art}

\subsubsection{Le problème de la sécurité}

Comme expliqué précédemment, un VPN est une technologie permettant de créer des connexions sécurisés à travers un réseau non sûr. Il y a plusieurs menaces dont une connexion VPN doit être capable de nous protéger : l'espionnage, l'altération des données et le rejeu de paquets.

L'espionnage de données correspond à l'aspect le plus connu auquel sont confrontés les échanges, allant jusqu'à occulter (à tord) les deux autres. La parade contre l'espionnage consiste en la mise en place d'un chiffrement fort des données.

L'interception de données correspond au cas où un attaquant est présent en tant qu'intermédiaire dans le flot de communication, capable d'insérer, de modifier ou de supprimer des paquets circulant dans ce flot. Cette problématique a été à la fois la plus importante et la plus difficile à résoudre, la parade consistant à une authentification systématique en signant chaque paquet émis afin de pouvoir identifier ceux qui sont frauduleux. Le mécanisme le plus répendu d'authentification de paquet est nommé construction HMAC (Hash Message Authentication Code).

Le problème du rejeu est complémentaire de l'interception : il s'agit de réémettre des paquets valides ayant déjà circulé sur le réseau avec bien sûr de mauvaises intentions. La parade consiste ici à inclure un identifiant unique (ou un timestamp) pour chaque paquet dans sa signature. Lorsqu'une entité reçoit un paquet elle garde trace des identifiants unique déjà reçus et refuse un paquet qui serait déjà passé. Des implémentations d'algorithmes à glissement de fenêtre sont les protections les plus utilisées contre le rejeu.

\subsubsection{Le problème de l'authentification}

L'authentification des tiers est l'une des solutions clefs pour mettre en place des communications sécurisées. Néanmoins ce moyen de résoudre un problème ne fait que conduire à un autre plus gros encore : celui de la gestion des clefs. En effet les algorithmes utilisés pour le chiffrement des échanges réclament des clefs de session pour fonctionner, et pour que notre connexion puisse être considérée comme sécurisée ces clefs doivent changer régulièrement. Ainsi la mise en place d'une connexion sécurisée impose d'avoir un moyen d'échanger des clefs de session de façon sécurisée alors que notre connexion ne l'est pas encore : c'est la cryptographie à clefs privées qui va résout ce problème.



\subsubsection{Les différents types de VPN}

Il existe deux façons d'aborder la mise en place d'une technologie VPN, chacune ayant sa finalité : les VPN bridgés et les VPN routés. Les VPN bridgés sont généralement utilisés lors de connexions site-à-site, tandis que les VPN routés sont eux préférés pour connecter des clients nomades. La mise en place d'un VPN routé s'impose donc pour cette étude, car en plus d'être facilement intégrable à l'architecture existante, elle est parfaitement adaptée à l'usage que l'ISIMA pourrait en faire.

\subsubsection{Les classes de protocoles}
\paragraph{IPsec}
~

IPsec est une suite protocolaire de niveau 3, visant à apporter la sécurité manquant au protocole IP. Cette suite utilise plusieurs protocoles au cours des différentes phases de mise en place d'IPsec.

La première phase est une phase négociation au cours de laquelle les parties se mettent d'accord sur les algorithmes de chiffrement à utiliser et échangent des clefs de session via le protocole ISAKMP (Internet Security Association and Key Management Protocol). Au cours de cette phase le protocole IKE (Internet Key Exchange) intervient également pour générer ces clefs de session soit grâce à une clef partagée soit à l'aide de certificats RSA.

La seconde phase est la phase de communication au cours de laquelle les données peuvent traverser le tunnel et sont chiffrées. Deux protocoles peuvent intervenir en fonction de la finalité du tunnel : ESP qui fournit à la fois intégrité et confidentialité des données, et AH qui fournit l'intégrité et l'authentification.


\paragraph{pptp}
~

PPTP (Point to Point Tunneling Protocol) est un protocole dont le rôle consiste à construire des paquets PPP (Point to Point Protocole) pour les encapsuler dans des datagrammes IP. PPTP tire ainsi aventage des mécanismes d'authentification, de chiffrement et de compression déjà existants pour PPP (niveau 2) en les appliquant au niveau 3. Le principal aventage de ce protocole est son excellente intégration avec les systèmes Microsoft, au sein desquels la suite protocolaire de PPP a été réimplémentée pour accompagner PPTP.

Ainsi on utilise MS-Chap v2 (Microsoft-Challenge Handshake Authentication Protocol) pour l'authentification, et MPPe 'Microsoft Point to Point Encryption) pour le chiffrement. Au cours d'une session VPN entre un client et le serveur, une connexion TCP est utilisée pour le contrôle de la liaison. Quand à l'échange de données, celui-ci requiert un canal UDP et s'appuie sur le protocole GRE (Generic Routing Encapsulation).


\paragraph{TLS}
~

Le protocole TLS est une évolution du protocole SSL qui, après s'être rendu extrêmement populaire dans le domaine des transaction sécurisées au niveau applicatif (notamment le web), a être utilisé dans le domaine des VPN. Le but de ces technologies VPN est de s'appuyer sur la maturité de TLS pour gérer la gestion du tunnel de données ainsi que tous les éléments cryptographiques nécessaires : authentification, confidentialité et intégrité.

Une erreur commune est de penser que comme TLS n'est pas un protocole de niveau 3 les technologies s'appuyant sur lui ne répondent pas aux critères fondamentaux des VPN : il n'en est rien. Il s'agit simplement d'une approche différente du problème, issue d'une constatattion simple : Les systèmes complexes sont les plus difficiles à sécuriser. Là où la mise en place d'IPsec dépend d'une nouvelle pile IP et implique des intéractions à la fois très fortes et très sophistiquées avec le noyau du système, les technologies basées sur TLS s'appuient sur du code s'exécutant dans l'espace utilisateur : une simple interface réseau virtuelle est utilisée pour parvenir à ce résultat.


\subsection{Objectifs fixés}

Les objectifs de la suite de cette étude sont donc d'étudier la mise en place d'une plate-forme de test pour les différentes solutions qui ont été recensées conformément aux représentations logique et physique de l'architecture présentées en partie \ref{section_architecture_etudiee} page \pageref{section_architecture_etudiee}. Les différentes solutions seront finalement confrontées les uns aux autres dans une dernière partie.

Nous commencerons par étudier la solution Microsoft, viendra ensuite le logiciel OpenVPN sous Linux, et enfin la technologie CISCO. Pour la mise en place de chacune de ces trois solutions, nous nous attarderons sur les problématiques de l'installation et de la configuration, que ce soit côté serveur ou côté client. Chacune de ces parties fera état des problèmes rencontrés au cours de l'étude ainsi que des limites de chaque solution.



\pagebreak

	\section{Mise en place des maquettes}

	Dans cette partie, nous allons détailler les trois architectures mise en place pour la création d'un VPN.Nous commenceront par détailler la solution VPN que Microsoft propose.

\subsection{Solution Windows}



	Pour ce projet, nous avons décider d'installer Windows Server 2003 Entreprise Edition, afin d'avoir un sytème d'exploitation récent. Toutes les fonctionnalités utilisées pour la création d'un VPN sont incluses dans cette version. Il n'y a pas besoin d'installer une application tierce.
	Voyons à présent les différents services nécessaires pour la mise en place du VPN de Microsoft.

\subsubsection{Services installés}

	Afin de pouvoir se mettre dans les conditions du réseau interne de l'ISIMA, nous avons décidé de configurer plusieurs services:
~


\begin{itemize}
	\item  DHCP,
	\item  DNS, 
	\item  Active Directory,
	\item  Routage et accès distant (VPN),
	\item  Autorité de certification (CA).
\end{itemize}
~	

	Ces nombreux services sont nécessaires pour la création d'un VPN. A présent, détaillons la configuration de chacun de ces servives. 
	Remarque : Sous windows on ne parle pas de serveur mais de service. Tout les fonctionnalités de la machine (DNS, DHCP ou autre) sont référencés comme des services.


\paragraph{Caractéristique de la machine}
~\

Le serveur windows est installé sur un DELL PowerEdge 1300. Il est doté d'un processeur INTEL de type Pentium II cadencé à 348 Mhz. Il possède 512 Mo de mémoire vive et un disque dur de 9 Go.
Le serveur est composé de trois cartes réseaux, une pour le réseau extérieur et deux autres pour le réseau interne. 



Voici leur adresse IP:

\begin{figure}[H]
	\begin{center}
\begin{tabular}{|l|c|c|}
\hline
Caractéristique & Réseau & Adresse IP \\
\hline
Realtek RTL 8139 Familly & réseau PROFS & 10.0.0.11/24 \\
Realtek RTL 8139 Familly & réseau extérieur & 192.168.102.250/24 \\
Fast Ethernet CNET PRO 200P & réseau STUDENTS & 192.168.0.11/24 \\
\hline
\end{tabular}
	\end{center}
	\caption{Caractéristique des cartes réseaux}
	\label{IP_carte_reseau}
\end{figure}

Tout les cartes installés sont de type 100Mbit/s.
~

\paragraph{Service DHCP}
~\


Ce service est nécessaire pour l'attribution des adresses IP qui seront allouées aux différents clients lors de leurs connexion au serveur VPN. Nous avons configurer le service DHCP de la manière suivante. Sachant que le serveur VPN doit avoir deux cartes réseaux indépendantes (une sur le réseau STUDENTS et une sur le réseau PROFS) le DHCP peut fournir deux pools d'adresse différents. 
Voici les caractéristiques des deux réseaux:

\begin{figure}[H]
	\begin{center}
\begin{tabular}{|l|c|c|}
\hline
Caractéristique & Réseau PROFS & Réseau STUDENTS \\
\hline
Désignation Carte réseau & DHCP\_PROFS & DHCP\_STUDENTS \\
Adresse IP Carte réseau & 10.0.0.11 & 192.168.0.11 \\
Pool d'adresse & 10.0.2.100 à 10.0.2.254 & 192.168.0.100 à 192.168.0.254 \\
Masque réseau & 255.255.255.0 & 255.255.255.0 \\
Serveur DNS/routeur & 10.0.0.11 & 192.168.0.11 \\
Nom du domaine DNS & wvpn.isima.fr & wvpn.isima.fr \\
\hline
\end{tabular}
	\end{center}
	\caption{Caractéristique du service DHCP}
	\label{service_DHCP}
\end{figure}

Il est à noter que lors de l'installation du service DHCP, et en dehors des différentes configurations des cartes, il n'y a pas d'options spécifiques. Ce sont les options par défaut.
Il est à remarquer que le service DHCP se compose de deux étendues, une pour chaque carte. Cependant, le service DHCP s'attache à une seule carte réseau et non au deux. Dans la maquette, le service se lie avec la carte DHCP\_PROFS. Pour illustrer ces propos, voici un screenshot du service DHCP.

\begin{figure}[H]
	\begin{center}
	\includegraphics[width=\textwidth]{partie_2/screen_windows/dhcp.JPG}\\
	\end{center}
	\caption{Lien avec la carte réseau DHCP\_PROFS}
	\label{Screen_client_dhcp}
\end{figure}


On remarque que bien que la présence des deux étendues et que la carte DHCP\_PROFS (10.0.0.11) s'attache au service DHCP.
~\


Voyons à présent la configuration du service DNS.

\paragraph{Service DNS}
~\


Afin de pouvoir utiliser l'active directory nous avons du installer ce service. Celui-ci nous permet de résoudre les différentes adresses IP qui seront nécessaire lorsq'un client VPN se connectera sur le réseau interne.
Voici les caractèristiques du service DNS:
~\


\begin{figure}[H]
	\begin{center}
\begin{tabular}{|l|c|c|}
\hline
Caractéristique & Serveur Windows2003 \\
\hline
Nom de la machine &  win2k3vpn\\
Nom du domaine & wvpn.isima.fr\\
Type de zone &  zone principale\\
\hline
\end{tabular}
	\end{center}
	\caption{Caractéristique du service DNS}
	\label{service_DNS}
\end{figure}

~\

Il est à remarquer que lors de l'installation du service, nous avons décider d'incorporer la zone de recherche inversée. Le serveur windows étant composé de trois cartes réseaux, il y a trois zones. Afin de vérifier que le service DNS est bien configurer, la commande NSLOOKUP nous permet de tester le DNS. 
L'illustration suivante montre la résolution DNS est correcte.

\begin{figure}[H]
	\begin{center}
		\includegraphics[width=\textwidth]{partie_2/screen_windows/nslookup.JPG}\\
	\end{center}
	\caption{Résolution DNS}
	\label{NSLOOKUP}
\end{figure}

Tout comme le service DHCP, le service DNS se lie avec une seule carte réseau. 

Intéressons nous à présent à la configuration de l'Active Directory.

\paragraph{Active Directory}
~\

Ce service nous a permi de créer un annuaire afin d'indentifier les clients souhaitant se connecter sur le réseau interne. Afin de pouvoir effectuer des tests sur la connexion VPN, nous avons dans un premier temps créeer deux groupes: un groupe STUDENTS et un groupe PROFS.

Il est à noter que lorsqu'on crée un utilisateur, on doit l'ajouter dans le groupe PROFS ou STUDENTS sans oublier de lui donner l'autorisation de pouvoir se connecter. 

\begin{figure}[H]
	\begin{center}
		\includegraphics[width=0.50\textwidth]{partie_2/screen_windows/vpn.JPG}\\
	\end{center}
	\caption{Autorisation de se connecter via VPN sur le serveur Windows 2k3}
	\label{VPN_AUTORISATION}
\end{figure}

Du point de vue de la stratégie de groupe (GPO) nous avons décider de laisser la configuration par défaut.

Une fois les services de base installé, voyons à présent la configuration du service de \textit{Routage et accès distant} .

\paragraph{Routage et accès distant}
~\


Le service routage et accès distant est le service qui va nous permettre de monter un tunnel sécurisé. Lors de l'installation du service, l'administrateur doit choisir l'option ``routage pour réseaux locaux uniquement'' ainsi que l'option ``serveur d'accès distant''. Une fois le service installé, il faut configurer les options de sécurité qui sont propre au serveur (un clique droit sur le nom du serveur permet de modififer les propriétés). Cela correspond à la manière que les utilisateurs s'identifie. Le service propose deux choix : identification par l'annuaire ou bien en utilisant un serveur RADIUS.

\begin{figure}[H]
	\begin{center}
		\includegraphics[width=0.50\textwidth]{partie_2/screen_windows/secu_vpn.PNG}\\
	\end{center}
	\caption{Choix de la méthode d'authentification}
	\label{VPN_AUTHENTIFICATION}
\end{figure}

Nous avons essayer de tester la maquette en utilisant le serveur RADIUS afin que celui-ci authentifie l'utilisateur avec la base NIS. Cependant les essais n'ont pas été concluant, nous n'avons pas pu nous identifier en utilisant notre compte ISIMA. Afin de rendre la maquette fonctionnel, nous avons dû laisser l'authentification par l'intermédiaire de l'annuaire local.

Dans l'onglet IP, il faut configurer la manière dont le service va distribuer les adresses IP. Ici nous avons le choix, soit le système gère les adresses IP en utilisant le service DHCP, soit l'administrateur peut remplir manuellement la plage d'adresse IP. La dernière option ``carte'' qui correpond à un menu déroulant nous permettant de selectionner la carte réseau qui repondrera aux demandes de la création d'un tunnel sécurisé. 

\begin{figure}[H]
	\begin{center}
		\includegraphics[width=0.50\textwidth]{partie_2/screen_windows/choix_carte.PNG}\\
	\end{center}
	\caption{Selection de l'interface réseau}
	\label{VPN_CARTE_ECOUTE}
\end{figure}

En naviguant dans les différentes options, on se rend rapidement compte que le service d'accès distant se lie avec une seule carte réseau. C'est la grande limitation. Il n'est pas possible que les deux cartes réseaux destinés pour le traffic interne redirige les différents flux réseaux.

D'un point de vue sécurité, il est possible de créer des stratégies d'accès distant. Pour le projet, nous avons créer une stratégie avec les spécifications suivantes:
~\
\begin{itemize}
 	\item Appartenance à un groupe : PROFS ou STUDENTS
	\item Type d'authentification : MS-CHAP V2
	\item Spécification du type de connexion : VPN
\end{itemize}
~


Cette stratégie d'accès distant comporte une priorité de 1. Si un client respecte tous ces critères il pourra alors se connecter sur le serveur VPN.

\begin{figure}[H]
	\begin{center}
		\includegraphics[width=0.50\textwidth]{partie_2/screen_windows/strat.png}\\
	\end{center}
	\caption{Stratégie d'accès distant}
	\label{VPN_STRAT}
\end{figure}

Dans la stratégie d'accès on peut rajouter un paramètre sur la date. En effet, on le paramétrant, on peut définir une plage horaire où l'utilisateur ne pourra pas se connecter.

Concernant le tyoe d'authentification, on utilise le protocole MS-CHAP V2 qui est un protocole propriétaire de Miscrosoft. En complement de cela, le serveur VPN fait un challenge de type MD5 afin d'authentifier le client.

Le protocole MS-CHAP V2 utilise une authentification mutuelle. Cela permet au serveur d'authentification et à la machine distante de vérifier leurs identités respectives. L'inconvenient de cette méthode est que le login du client passe en clair sur le réseau. En sniffant la connexion par l'intermédiare de Wireshark, nous nous sommes rendu compte de cela. Par contre, le challenge MD5, et les mots de passe sont chiffrés.



Afin d'améliorer la sécurité, nous avons décider de mettre en place une autorité de certification.

\paragraph{Autorité de certification}

Cette autorité va nous permettre de délivrer des certificats qui nous seront utile par la suite dans la configuration du routeur CISCO. Pour la configuration de l'autorité nous avons choisi une autorité racine autonime ayant comme nom commun ISIMA. Les certificats ont une validité d'un an.

\subsubsection{Solution dans son fonctionnement}
\subsubsection{Limites de la solution}

\subsection{Solution Linux}
\subsubsection{Configuration du VPN}
\subsubsection{Sécurité+protocoles}
\subsubsection{Limites de la solution}

\subsection{OpenVPN}

\subsubsection{Généralités}
OpenVPN est un outil Open-Source permettant de créer des tunnels sécurisés (SSL/TLS) à travers un réseau non sûr comme Internet. OpenVPN est à la fois facile à installer et à configurer, en plus d'être disponible sur à peut prêt toutes les plates-formes (Linux, Windows, BSD, Mac OS, Solaris). Le principe de configuration reste le même quelque soit la plate-forme utilisée.

OpenVPN est basé sur une architecture client-server. Le VPN fonctionne soit site-à-site, soit avec des clefs partagées. Les données sont tunnelisées sur un seul port, TCP ou UDP.

\subsubsection{Les deux types de VPN}
OpenVPN propose deux types de VPN : les VPN bridgés et les VPN routés. Dans le cas du mode bridgé le réseau virtuel créé devient une réelle extension du réseau local. L'avantage de ce mode est la facilité d'intégration de la solution VPN au sein de l'infrastructure déjà en place. Ce mode est d'ailleurs la seule option si pour une raison ou pour une autre des paquets broadcasts doivent traverser le VPN. Le principal inconvénient de ce mode apparait lors du passage à l'échelle : comme c'est le réseau local qui doit absorber les clients du VPN, il faut suffisament de ressources disponibles (adresses IP, etc.).

TODO : c'est pas tout à fait vrai

Le mode routé est le mode le plus utilisé. Bien que sa mise en place soit plus complexe, ce mode permet de faire du réseau virtuel un réseau à part du réseau local. On est ainsi capable de mettre en place une politique d'accès différente pour les utilisateurs connectés depuis l'extérieur, ce qui renforce encore la sécurité de l'infrastructure. Le second grand aventage des VPN routés est que le passage a l'échelle s'effectue ne pose aucun problème étant donné que l'on n'impacte pas l'utilisation du réseau local.

~

La figure \ref{tableau_types_vpn} résume les avantages et inconvénient de chacun des deux types de VPN :
\begin{figure}[H]
	\begin{center}
		\begin{tabular}{c|c}
			VPN bridgé & VPN routé \\
			\hline
			extension du réseau local & réseau à part \\
			mauvais passage à l'échelle & passage à l'échelle \\
			laisse passer les broadcasts & broadcasts impossibles \\
		\end{tabular}
	\end{center}
	\caption{Avantages et inconvénients des deux types de VPN}
	\label{tableau_types_vpn}
\end{figure}

Nous avons choisi de mettre en place une configuration de VPN routé, car mieux adaptée à l'utilisation que l'ISIMA pourrait en faire.

\subsubsection{Mise en place côté serveur}

La mise en place de l'infrastructure s'effectue en plusieures étapes. Tout d'abord nous installerons et configurerons OpenVPN sur le serveur, ensuite nous génèrerons les clefs et certificats de sécurité nécessaires, et enfin nous mettrons en place les outils nécessaires pour authentifier les utilisateurs via l'annuaire NIS de l'ISIMA.

Les interfaces de la machine sont configurées comme indiqué sur le schéma ref{TODO}, et résumé dans le tableau figure \ref{linux_interfaces} :

\begin{figure}[H]
	\begin{center}
		\begin{tabular}{c|c|l}
			Interface & Adresse IP & Commentaire \\
			\hline
			eth0 & 192.168.0.10 & Réseau interne étudiants \\
			eth1 & 192.168.102.121 & Réseau externe \\
			eth2 & 10.0.0.10 & Réseau interne profs \\
		\end{tabular}
	\end{center}
	\caption{Configuration des interfaces de la machine Linux}
	\label{linux_interfaces}
\end{figure}


\subsubsection{Installation et configuration d'OpenVPN}

\paragraph{Résolution des dépendances}
~

La machine fonctionne sous \texttt{Linux CentOS 5.1}. OpenVPN n'étant pas disponible directement dans les paquets de cette distribution (datant d'il y a presque deux ans à l'écriture de ces lignes), nous allons construire notre propre rpm. Les paquets requis pour mener à bien cette étape sont à installer grâce à la commande suivante :

\verb|[root@centosvpn ~]# yum install openssl-devel pam-devel rpm-build gcc-c++|

~

La version d'OpenVPN utilisée pour le projet est la 2.0.9 disponible sur \verb|http://openvpn.net/|. Celle-ci dépend des paquets \verb|lzo-devel-1.08-fr2.i386| et \verb|lzo-1.08-fr2.i386|, disponibles par exemple sur \verb|http://rpmfind.net/|.

\verb|[root@centosvpn ~]# wget ftp://rpmfind.net/linux/freshrpms/redhat/9/|

\verb|[root@centosvpn ~]# lzo/lzo-devel-1.08-fr2.i386.rpm|

\verb|[root@centosvpn ~]# rpm -ivh lzo-1.08-fr2.i386 lzo-devel-1.08-fr2.i386.rpm|

\paragraph{Compiler OpenVPN}
~

Lors de la configuration de notre maquette la version courante d'OpenVPN était la 2.0.9; adaptez les numéros de version avec celui de la dernière release stable d'OpenVPN. Les commandes suivantes permettent à la fois de la récupérer, de la compiler, et de l'installer :

\verb|[root@centosvpn ~]# wget http://openvpn.net/release/openvpn-2.0.9.tar.gz|

\verb|[root@centosvpn ~]# rpmbuild -tb openvpn-2.0.9.tar.gz|

\verb|[root@centosvpn ~]# rpm -ivh /usr/src/redhat/RPMS/i386/openvpn-2.0.9-1.i386.rpm|

\paragraph{Configuration de base}
~

Une instance d'OpenVPN ne peut gérer qu'un seul pool d'adresses à la fois, et donc un seul type de clients pour le VPN. La solution pour gérer à la fois les professeurs et les étudiants grâce à une même machine est donc de lancer deux instances du serveur en écoute sur un port différent. Nous allons donc utiliser deux fichiers de configuration distincts dont la figure \ref{configuration_base_openvpn} présente les paramètres qui leurs sont communs : Interface d'écoute, protocole de transport, etc.

\begin{figure}[H]
	\begin{center}
		\begin{minipage}{0.90\textwidth}
			\begin{lstlisting}[frame=trBL]
local 192.168.102.121
proto udp
dev tap
client-to-client
duplicate-cn
keepalive 10 120
comp-lzo
user nobody
group nobody
persist-key
persist-tun
status openvpn-status.log
verb 3
			\end{lstlisting}
		\end{minipage}
	\end{center}
	\caption{Configuration de base d'OpenVPN}
	\label{configuration_base_openvpn}
\end{figure}

~

Pour configurer correctement deux instances qui puissent cohabiter, celles-ci doivent se mettre en écoute sur un port différent. Nous avons choisi le port 1194 (port par défaut d'OpenVPN) pour le serveur profs, ainsi que le port 1195 pour le serveur étudiant. Les figures \ref{configuration_base_prof} et \ref{configuration_base_student} détaillent également la configuration des pool d'adresses à affecter aux clients, ainsi que les informations de routage à leur transmettre :

\begin{figure}[H]
	\begin{lstlisting}[frame=trBL]
port 1194
server 10.0.1.0 255.255.255.0
ifconfig-pool-persist ipp-profs.txt
push "route 10.0.0.0 255.255.255.0"
	\end{lstlisting}
	\caption{Configuration spécifique à \texttt{/etc/openvpn/server-prof.conf}}
	\label{configuration_base_prof}
\end{figure}
\begin{figure}[H]
	\begin{lstlisting}[frame=trBL]
port 1195
server 192.168.1.0 255.255.255.0
ifconfig-pool-persist ipp-profs.txt
push "route 192.168.0.0 255.255.255.0"
	\end{lstlisting}
	\caption{Configuration spécifique à \texttt{/etc/openvpn/server-student.conf}}
	\label{configuration_base_student}
\end{figure}



% ca /etc/openvpn/ca.crt
% cert /etc/openvpn/openvpn.crt
% key /etc/openvpn/openvpn.key  # This file should be kept secret
% dh /etc/openvpn/dh1024.pem
% tls-auth /etc/openvpn/ta.key 0 # This file is secret
% plugin /usr/share/openvpn/plugin/lib/openvpn-auth-pam.so login

\subsubsection{Génération des clefs et certificats de sécurité}

Nous allons maintenant utiliser openssl pour générer nos clefs et certificats de sécurité.

\subsubsection{Authentification via l'annuaire de l'ISIMA}

OpenVPN étant capable de réaliser une authentification PAM (méthode standart sur les systèmes UNIX), c'est la solution que nous avons retenue. Nous allons donc configurer un client NIS sur la machine de test, et indiquer à OpenVPN comment l'utiliser.

La première étape consiste à installer le client NIS si ce n'est pas déjà fait (paquet \texttt|ypserv|) et à faire entrer la machine dans le domaine NIS de l'ISIMA. \texttt{glouglou.isima.fr}.

\subsubsection{Configuration de démarrage de la machine}


\pagebreak


% \begin{figure}[H]
% 	\begin{center}
% 		\begin{minipage}{0.90\textwidth}
% 			\begin{lstlisting}[frame=trBL]
% 
% 			\end{lstlisting}
% 		\end{minipage}
% 	\end{center}
% 	\caption{Récupération des éléments de réponse dans un message de commande}
% 	\label{format_reponse_commande}
% \end{figure}


\subsection{Solution Cisco}
\subsubsection{Protocoles utilisés}
\subsubsection{Système d'authentification}
\subsubsection{Configuration du routeur}
\subsubsection{Limites de la solution}

\subsection{Solution Cisco}

\subsubsection{Généralités}
ipsec esp-sha1

\subsubsection{Mise en place côté serveur}


\paragraph{Configuration de base}
~\\

Commençons par configurer les interfaces du routeur. L'interface connectée au réseau de l'ISIMA récupère son adresse IP via DHCP. Cela :
\begin{figure}[H]
	\begin{center}
		\begin{minipage}{0.90\textwidth}
			\begin{lstlisting}[frame=trBL]
(config)# hostname CISCOVPN
(config)# enable secret cisco
(config)# no ip domain-lookup
(config)# interface FastEthernet0/0
(config-if)# description interface to the external network
(config-if)# ip address dhcp
(config-if)# no shutdown
(config-if)# exit
(config)# interface FastEthernet0/1.1
(config-if)# description interface to the prof network
(config-if)# encapsulation dot1Q 333
(config-if)# ip address 10.0.0.1 255.255.255.0
(config-if)# exit
(config)# interface FastEthernet0/1.2
(config-if)# description interface to the student network
(config-if)# encapsulation dot1Q 111
(config-if)# ip address 192.168.0.1 255.255.255.0
(config-if)# exit
(config)# interface FastEthernet0/1
(config-if)# no shutdown
(config-if)# exit
			\end{lstlisting}
		\end{minipage}
	\end{center}
	\caption{Configuration des interfaces}
	\label{configuration_interfaces}
\end{figure}

~

Configurons maintenant les accès à distance au routeur :
\begin{figure}[H]
	\begin{center}
		\begin{minipage}{0.90\textwidth}
			\begin{lstlisting}[frame=trBL]
(config)# banner login #Unauthorized access prohibited - F5 only!#
(config)# banner motd #
This router is part of a wonderfull ZZ3F5 project for 2008-2009.
If you have any question, comment, insults, whatsoever...
please contact coscia@poste.isima.fr and dessaux@poste.isima.fr.
Thank you if you read this till the end.#
(config)# enable secret cisco
(config)# line con 0
(config-line)# logging synchronous
(config-line)# password cisco
(config-line)# login
(config-line)# exit
(config)# line vty 0 4
(config-line)# transport input telnet
(config-line)# password cisco
(config-line)# login
(config-line)# exit
(config)# service password-encryption
			\end{lstlisting}
		\end{minipage}
	\end{center}
	\caption{Configuration des interfaces}
	\label{configuration_interfaces}
\end{figure}

\paragraph{Configuration de l'authentification radius}
~\\

Tout d'abord le serveur FreeRadius.
Authentication PAP entre radius server and cisco router :
Ajouter \verb|ciscovpn User-Password := "isima"|  au début du fichier \verb|/etc/raddb/users|, où ciscovpn est le hostname du routeur et isima le mot de passe qui lui sera associé.

Ensuite le cisco.

\paragraph{Configuration du VPN IPSEC}
~\\

Les lignes qui suivent permettent de configurer la cryptographie isakmp :
\begin{itemize}
	\item algorithme de chiffrement triple DES.
	\item algorithme de hashage sha-1.
	\item authentification via clefs partagées.
	\item Diffie-Hellman 1024 bits.
	\item durée de vie du contexte cryptographique égale à une journée.
	\item utilisation du hostname plutôt que de l'adresse IP pour protéger les échanges.
\end{itemize}


\begin{figure}[H]
	\begin{center}
		\begin{minipage}{0.90\textwidth}
			\begin{lstlisting}[frame=trBL]
(config)# crypto isakmp policy 1
(config-isakmp)# encryption 3des
(config-isakmp)# hash sha
(config-isakmp)# authentication pre-share
(config-isakmp)# group 2
(config-isakmp)# lifetime 86400
(config-isakmp)# exit
(config)# crypto isakmp identity hostname
			\end{lstlisting}
		\end{minipage}
	\end{center}
	\caption{Configuration IKE}
	\label{configuration_ike}
\end{figure}

Ajoutons les pools DHCP qui fourniront leurs adresses IP aux étudiants et aux professeurs :
\begin{figure}[H]
	\begin{center}
		\begin{minipage}{0.90\textwidth}
			\begin{lstlisting}[frame=trBL]
(config)# ip local pool profs 10.0.1.20 10.0.1.254
(config)# crypto isakmp client configuration group profs
(config-isakmp-group)# key isimaprofs
(config-isakmp-group)# dns 10.0.0.11
(config-isakmp-group)# domain isima.fr
(config-isakmp-group)# pool profs
(config-isakmp-group)# exit
(config)# ip local pool students 192.168.1.20 192.168.1.254
(config)# crypto isakmp client configuration group students
(config-isakmp-group)# key isimastudents
(config-isakmp-group)# dns 192.168.1.11
(config-isakmp-group)# domain isima.fr
(config-isakmp-group)# pool students
(config-isakmp-group)# exit
			\end{lstlisting}
		\end{minipage}
	\end{center}
	\caption{Configuration des interfaces}
	\label{configuration_interfaces}
\end{figure}

% TODO certificats

\subsubsection{Mise en place côté client}




\pagebreak

	\section{Confrontation des résultats}

Dans cette partie, nous allons confronter l'ensemble des résultats obtenus afin de choisir la solution correspondant le plus au cahier des charges fixé en première partie de ce rapport. Nous commencerons par mettre en évidence les critères selon lesquels les différentes solutions seront évaluées, avant d'effectuer la confrontation elle-même, pour finalement établir un bilan.

\subsection{Critères retenus}

Avant de définir les différents critères que lequelle nous nous sommes basé, un petit rappel du cahier des charges nous semble judicieux.


\begin{itemize}
 	\item implémentation rapide côté serveur,
	\item le client VPN doit être utilisable sur n'importe quelle plateforme,
	\item la configuration du client VPN doit être simple et rapide,
	\item s'adapter aux différents serveurs de l'ISIMA.
\end{itemize}


En fonction de ce cahier des charges, nous avons défini plusieurs critères, qui seront répartis dans trois sous parties. Commençons par les critères du côté client. 

\subsubsection{Côte Client}

\begin{center}
% use packages: array,longtable
\begin{longtable}{lll}
 & × & × \\ 
× & × & × \\ 
× & × & ×
\end{longtable}
\end{center}


\subsubsection{Côté Serveur}

\subsubsection{Securité}

\subsection{Analyse des performances}

Dans cette section nous allons mener une analyse des performances de chaque solution, en termes de quantité de trafic que les machines sont capables de traiter. En raison de l'utilisation massive des algorithmes de chiffrement et déchiffrement, ce sont les capacités processeur des machines seront déterminantes.

Nous avons utilisé un outil Open-Source afin de mener à bien cette évaluation des performances. Ce logiciel très simple nommé IPperf envoie simplement le plus de paquets d'un point à un autre.

\subsubsection{Configuration d'IPperf}

Le but de ce benchmark est de générer du trafic et de forcer son passage à travers le tunnel VPN. Pour pouvoir effectuer des mesures, IPperf doit être lancé sur une machine cliente d'un côté du tunnel, ainsi que sur une machine serveur de l'autre côté :

\begin{figure}[H]
	\begin{center}
		\includegraphics[width=0.75\textwidth]{partie_3/images/ipperf.png}\\
	\end{center}
	\caption{Protocole de test}
	\label{Protocole_de_test}
\end{figure}

Voici la configuration d'IPperf :
En mode serveur : \verb|iperf.exe -s -i1|. L'option \verb|-s| correspond au mode serveur et le \verb|-i1| correpond à la fréquence de prise de mesure (ici une seconde).

En mode client : \verb|iperf.exe -c 10.0.1.25 -i1 -t40 |. L'option \verb|-c| correspond au mode client et l'option \verb|-t40| indique la durée pendant laquelle on va générer du trafic.

\subsubsection{Confrontration des résultats}

Voici les caractéristiques de la plateforme ayant fait office de serveur :
\begin{figure}[H]
	\begin{center}
\begin{tabular}{l|c}
Caractéristiques & Machine \\
\hline
CPU & Intel Core 2 Duo T8100 2,10GHz \\
RAM & 3 Go \\
Carte réseau & Intel 82566MM Gigabit \\
\end{tabular}
	\end{center}
	\caption{Caractéristiques de la plateforme serveur}
	\label{Caractéristique_de_la_plateforme_serveur}
\end{figure}

Les machines clientes utilisées sont celles présentes en salle A214. La figure \ref{Résultat_des_benchmarks} présente les résultats que nous avons obtenus. Le débit indiqué correspond à une moyenne sur quarante secondes :

\begin{figure}[H]
	\begin{center}
\begin{tabular}{l|c|c|c}
Solutions & WINDOWS & LINUX & CISCO \\
\hline
Bande Passante(moyenne) & 18Mbps & 50Mbps & 31Mbps \\
Utilisation CPU & 100\% & 80\% & 100\% \\
\end{tabular}
	\end{center}
	\caption{Résultats des benchmarks}
	\label{Résultat_des_benchmarks}
\end{figure}

Le taux d'utilisation processeur n'est pas fournit par IPperf. Cette valeur a été obtenue directement en visualisant la charge de la machine. Dès lors que du trafic traverse le tunnel, les serveurs VPN se trouvent fortement sollicités, et il devient rapidement très difficile d'exécuter une tâche en parallèle. Comme prévu, le taux d'occupation processeur est le facteur limitant pour chaque solution, ce dernier étant responsable du traitement des packets transitant sur le carte réseau.

En terme de bande passante, la solution OpenVPN se révèle être la plus performante. Cependant, nous avons remarqué des fluctuations durant les mesures, ce qui n'est pas le cas de routeur CISCO dont le débit est extrêment stable.

\subsection{Bilan}

% TODO inserer les différents graphes en étoile.



\pagebreak
%%%%%% Conclusion %%%%%%%%%%%%%%%%%%%%%%%%%%%%%%%%%%%%%%%%%%%%%%%%%%%%%%%%%%%%
	\fancyhead[R]{Conclusion}
	\section*{Conclusion}
\addcontentsline{toc}{section}{Conclusion}

~

L'objectif de ce projet était d'étudier la mise en place de trois solutions VPN. Nous avons commencé par étudier la solution de Microsoft, puis le logiciel OpenVPN sous Linux, et enfin la solution de CISCO. L'étude de ces solutions a été guidée par plusieurs critères constituant notre cahier des charges, comme la facilité de déploiement des clients et du serveur, le niveau de sécurité de l'accès et enfin la complexité de l'administration du serveur VPN.

~

Au cours de ce projet, nous avons rencontré trois grandes difficultés. La première concerne la problématique de la sécurité, car même si nous avons une formation théorique sur les différents protocoles et mécanismes qui y sont liés, cela reste insuffisant pour comprendre les enjeux de la sécurité. Ce projet a été pour nous l'opportunité d'appliquer nos bagages théoriques à un problème concret.

~

Une seconde difficulté concerne le peu de documentation sur ce sujet. En effet, dès que l'on commence à chercher des informations spécifiques, comme par exemple la configuration d'un service en particulier, ou sur la génération de clefs on ne trouve rien de concluant. Même sur les sites des différents constructeurs, l'informations n'est pas explicitement présentée.
Une dernière difficulté, concerne les limitations du système d'exploitation de Microsoft. En effet, il nous a fallu du temps pour comprendre que les services réseau se lient à une seule carte, même si leur fonctionnement doit en impacter plusieurs. Nous avons essayer de trouver divers moyens pour contourner ce problème mais sans aucun succès.

~

Dans un futur proche, nous pensons que nos différentes maquettes peuvent être améliorées. En effet, pour la solution CISCO il faudrait achever de mettre en place une authentification par certificats pour améliorer le niveau de sécurité. Concernant OpenVPN, on pourrait envisager d'implementer un renouvellement automatique des certificats chaque année, qui les publierait directement sur l'intranet de l'ISIMA.

~

Si l'ISIMA souhaite concrétiser son projet de solution VPN, nous recommandons l'implémentation d'OpenVPN car il est a parfaitement répondu au cahier des charges, se révélant à la fois performant, sécurisé et compatible avec n'importe quel type de système d'exploitation.

\pagebreak

%%%%%% Bibliographie %%%%%%%%%%%%%%%%%%%%%%%%%%%%%%%%%%%%%%%%%%%%%%%%%%%%%%%%%
	\fancyhead[R]{Bibliographie}
	\renewcommand{\bibname}{Bibliographie}
\renewcommand{\refname}{Bibliographie}

\begin{thebibliography}{Bibliographie}
\bibliographystyle{alpha}
\addcontentsline{toc}{section}{Bibliographie}
	\bibitem[1]{biblio_1} Mise en place d'un client RADIUS sous Windows [en ligne]. Etats-Unis. Disponible sur \verb|http://technet.microsoft.com/fr-fr/library/cc757473.aspx|
	\bibitem[2]{biblio_2} Installation du service VPN sous Windows [en ligne]. Etats-Unis. Disponible sur \verb|http://technet.microsoft.com/fr-fr/library/cc781701.aspx|
	\bibitem[3]{biblio_3} Documentation générale sur WindowsServer2k3 [en ligne]. Etats-Unis. Disponible sur \verb|http://technet.microsoft.com/fr-fr/library/cc706993.aspx|
	\bibitem[4]{biblio_4} Documentation de Windows Services for Unix [en ligne]. Etats-Unis. Disponible sur \verb|http://technet.microsoft.com/fr-fr/library/bb463193.aspx|
	\bibitem[5]{blblio_5} Service SCEP sous Windows [en ligne]. Etats-Unis. Disponible sur \verb|http://www.mi| \verb|crosoft.com/downloads/details.aspx?FamilyID=9f306763-d036-41d8-8860-16| \verb|36411b2d01\&DisplayLang=en|
	\bibitem[6]{biblio_6} Site officiel d'OpenVPN [en ligne]. Etats-Unis. Disponible sur \verb|http://openvpn.net/|
	\bibitem[7]{biblio_7} Document The User-Space VPN and OpenVPN [en ligne]. Etats-Unis. Disponible sur \verb|http://openvpn.net/papers/BLUG-talk/index.html|

	\bibitem[8]{biblio_8} CISCO IOS Cookbook, O'REILLY, Kevin Dooley \& Ian J. Brown, 2007.
	\bibitem[9]{biblio_9} Site officiel de CISCO [en ligne]. Etats-Unis. Disponible sur \verb|http://www.cisco.com|
\end{thebibliography}

\pagebreak

	\pagebreak
%%%%%% lexique %%%%%%%%%%%%%%%%%%%%%%%%%%%%%%%%%%%%%%%%%%%%%%%%%%%%%%%%%%%%%%%
	\fancyhead[R]{Lexique}
	\section*{Lexique}
\addcontentsline{toc}{section}{Lexique}

\begin{description}
	\item [Open-Source:] Se dit d'un logiciel dont la licence correspond à certains critères comme le libre accès à son code source ainsi que sa libre redistribution.
	\item [IPsec:] TODO
	\item [Protocole de transport:] Protocole dont le rôle consiste à délivrer les données aux applications.
	\item [Proxy:] Serveur jouant un rôle dans la sécurité des réseaux, servant intermédiaire aux transactions.
	\item [TCP:] Protocole de transport fiable au dessus de IP.
	\item [TLS:] Protocole de session sécurisé et reposant sur TCP.
	\item [UDP:] Protocole de transport de type \textit{best-effort} au dessus de IP.
	\item [Socket:] Interface logicielle permettant l'utilisation des ressources réseau sur une machine.
\end{description}

\pagebreak
%%%%%% table des figures %%%%%%%%%%%%%%%%%%%%%%%%%%%%%%%%%%%%%%%%%%%%%%%%%%%%%
	\fancyhead[R]{Table des figures}
	\addcontentsline{toc}{section}{Table des figures}
	\listoffigures
\end{document}
