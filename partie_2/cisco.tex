\subsection{Solution Cisco}

\subsubsection{Généralités}
ipsec esp-sha1

\subsubsection{Mise en place côté serveur}


\paragraph{Configuration de base}

Commençons par configurer les interfaces du routeur :
\begin{figure}[H]
	\begin{center}
		\begin{minipage}{0.90\textwidth}
			\begin{lstlisting}[frame=trBL]
(config)# hostname CISCOVPN
(config)# enable secret cisco
(config)# no ip domain-lookup
(config)# interface FastEthernet0/0
(config-if)# description interface to the external network
(config-if)# ip address dhcp
(config-if)# no shutdown
(config-if)# exit
(config)# interface FastEthernet0/1.1
(config-if)# description interface to the prof network
(config-if)# encapsulation dot1Q 333
(config-if)# ip address 10.0.0.1 255.255.255.0
(config-if)# exit
(config)# interface FastEthernet0/1.2
(config-if)# description interface to the student network
(config-if)# encapsulation dot1Q 111
(config-if)# ip address 192.168.0.1 255.255.255.0
(config-if)# exit
(config)# interface FastEthernet0/1
(config-if)# no shutdown
(config-if)# exit
			\end{lstlisting}
		\end{minipage}
	\end{center}
	\caption{Configuration des interfaces}
	\label{configuration_interfaces}
\end{figure}

Ajoutons

\paragraph{Configuration de l'authentification radius}

Tout d'abord le serveur FreeRadius.
Authentication PAP entre radius server and cisco router :
Ajouter \verb|ciscovpn User-Password := "isima"|  au début du fichier \verb|/etc/raddb/users|, où ciscovpn est le hostname du routeur et isima le mot de passe qui lui sera associé.

Ensuite le cisco.

\paragraph{Configuration du VPN IPSEC}

IKE
isakmp



\subsubsection{Mise en place côté client}


