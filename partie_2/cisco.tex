\subsection{Solution Cisco}

\subsubsection{Généralités}
ipsec esp-sha1

\subsubsection{Mise en place côté serveur}


\paragraph{Configuration de base}
~

Commençons par configurer les interfaces du routeur. L'interface connectée au réseau de l'ISIMA récupère son adresse IP via DHCP. Cela :
\begin{figure}[H]
	\begin{center}
		\begin{minipage}{0.90\textwidth}
			\begin{lstlisting}[frame=trBL]
(config)# hostname CISCOVPN
(config)# enable secret cisco
(config)# no ip domain-lookup
(config)# interface FastEthernet0/0
(config-if)# description interface to the external network
(config-if)# ip address dhcp
(config-if)# no shutdown
(config-if)# exit
(config)# interface FastEthernet0/1.1
(config-if)# description interface to the prof network
(config-if)# encapsulation dot1Q 333
(config-if)# ip address 10.0.0.1 255.255.255.0
(config-if)# exit
(config)# interface FastEthernet0/1.2
(config-if)# description interface to the student network
(config-if)# encapsulation dot1Q 111
(config-if)# ip address 192.168.0.1 255.255.255.0
(config-if)# exit
(config)# interface FastEthernet0/1
(config-if)# no shutdown
(config-if)# exit
			\end{lstlisting}
		\end{minipage}
	\end{center}
	\caption{Configuration des interfaces}
	\label{configuration_interfaces}
\end{figure}

~

Configurons maintenant les accès à distance au routeur :
\begin{figure}[H]
	\begin{center}
		\begin{minipage}{0.95\textwidth}
			\begin{lstlisting}[frame=trBL]
(config)# banner login #Unauthorized access prohibited - F5 only!#
(config)# banner motd #
This router is part of a wonderfull ZZ3F5 project for 2008-2009.
If you have any question, comment, insults, whatsoever...
please contact coscia@poste.isima.fr and dessaux@poste.isima.fr.
Thank you if you read this till the end.#
(config)# enable secret cisco
(config)# line con 0
(config-line)# logging synchronous
(config-line)# password cisco
(config-line)# login
(config-line)# exit
(config)# line vty 0 4
(config-line)# transport input telnet
(config-line)# password cisco
(config-line)# login
(config-line)# exit
(config)# service password-encryption
			\end{lstlisting}
		\end{minipage}
	\end{center}
	\caption{Configuration de l'accès à distance}
	\label{configuration_acces_a_distance}
\end{figure}

\paragraph{Configuration de l'authentification Radius}
~

Tout d'abord le serveur FreeRadius.
Authentication PAP entre radius server and cisco router :
Ajouter \verb|ciscovpn User-Password := "isima"|  au début du fichier \verb|/etc/raddb/users|, où ciscovpn est le hostname du routeur et isima le mot de passe qui lui sera associé.
Ajouter quelques lignes pour autoriser le router à interroger la base RADIUS dans le fichier /etc/raddb/clients.conf, et commenter la ligne shadow dans radiusd.conf

~


\begin{figure}[H]
	\begin{center}
		\begin{minipage}{0.90\textwidth}
			\begin{lstlisting}[frame=trBL]
(config)# aaa new-model
(config)# radius-server host 192.168.102.121 auth-port 1812
acct-port 1813 key isima
(config)# ip radius source-interface FastEthernet 0/0
(config)# aaa group server radius RadiusServer
(config-sg-radius)# radius-server host 192.168.102.121 auth-port
1812 acct-port 1813
(config-sg-radius)# exit
(config)# aaa authentication login default group RadiusServer
			\end{lstlisting}
		\end{minipage}
	\end{center}
	\caption{Configuration de l'authentification Radius}
	\label{configuration_authentification_radius}
\end{figure}


\paragraph{Configuration d'IPsec}
~

Les lignes qui suivent permettent de configurer la cryptographie isakmp :
\begin{itemize}
	\item algorithme de chiffrement triple DES.
	\item algorithme de hashage sha-1.
	\item authentification via clefs partagées.
	\item Diffie-Hellman 1024 bits.
	\item durée de vie du contexte cryptographique égale à une journée.
	\item utilisation du hostname plutôt que de l'adresse IP pour protéger les échanges.
\end{itemize}


\begin{figure}[H]
	\begin{center}
		\begin{minipage}{0.90\textwidth}
			\begin{lstlisting}[frame=trBL]
(config)# crypto isakmp policy 1
(config-isakmp)# encryption 3des
(config-isakmp)# hash sha
(config-isakmp)# authentication pre-share
(config-isakmp)# group 2
(config-isakmp)# lifetime 86400
(config-isakmp)# exit
(config)# crypto isakmp identity hostname
			\end{lstlisting}
		\end{minipage}
	\end{center}
	\caption{Configuration IKE}
	\label{configuration_ike}
\end{figure}

Ajoutons les pools DHCP qui fourniront leurs adresses IP aux étudiants et aux professeurs :
\begin{figure}[H]
	\begin{center}
		\begin{minipage}{0.90\textwidth}
			\begin{lstlisting}[frame=trBL]
(config)# ip local pool profs 10.0.1.20 10.0.1.254
(config)# crypto isakmp client configuration group profs
(config-isakmp-group)# key isimaprofs
(config-isakmp-group)# dns 10.0.0.11
(config-isakmp-group)# domain isima.fr
(config-isakmp-group)# pool profs
(config-isakmp-group)# exit

(config)# ip local pool students 192.168.1.20 192.168.1.254
(config)# crypto isakmp client configuration group students
(config-isakmp-group)# key isimastudents
(config-isakmp-group)# dns 192.168.1.11
(config-isakmp-group)# domain isima.fr
(config-isakmp-group)# pool students
(config-isakmp-group)# exit
			\end{lstlisting}
		\end{minipage}
	\end{center}
	\caption{Configuration des pools utilisateurs}
	\label{configuration_pools_utilisateurs}
\end{figure}

% TODO certificats

Configurons maintenant la police IPsec :
\begin{itemize}
	\item Mise en place de l'ACL pour indiquer que l'on veut filtrer l'ensemble du trafic IP.
	\item Encapsulation ESP, chiffrement 3des, intégrité sha-1.
	\item Mode tunnel (ie au niveau 3, par opposition au mode transport au niveau 4).
	\item Configuration de l'authentification des profs et des étudiants via le serveur radius.
\end{itemize}

\begin{figure}[H]
	\begin{center}
		\begin{minipage}{1\textwidth}
			\begin{lstlisting}[frame=trBL]
(config)# access-list 101 permit ip any any
(config)# crypto ipsec transform-set policy esp-3des esp-sha-hmac
(cfg-crypto-trans)# mode tunnel
(cfg-crypto-trans)# exit
(config)# crypto dynamic-map prof-map 1
(config-crypto-map)# set transform-set policy
(config-crypto-map)# exit

(config)# crypto map prof-map
(config)# crypto map prof-map 1 ipsec-isakmp dynamic prof-map
(config)# crypto map prof-map client authentication list RadiusServer
(config)# cryto map prof-map client configuration address respond
(config)# crypto map prof-map isakmp authorization list 101

(config)# aaa authorization network 101 local
			\end{lstlisting}
		\end{minipage}
	\end{center}
	\caption{Configuration de la police IPsec}
	\label{configuration_police_ipsec}
\end{figure}

une seule crypto map, identification en fonction du group dans le client (appel direct au pool car la crypto map n'y est pas liée, ça marche). Le problème c'est qu'un étudiant peut facilement se faire passer pour un prof.

\begin{figure}[H]
	\begin{center}
		\begin{minipage}{1\textwidth}
			\begin{lstlisting}[frame=trBL]
(config)# crypto ca identity isima.fr
(ca-trustpoint)# enrollment url
http://192.168.102.250/certsrv/mscep/mscep.dll
(ca-trustpoint)# exit
(config)# crypto ca authenticate isima.fr
(config)# crypto ca trustpoint isima.fr
(ca-trustpoint)# crl optional
(ca-trustpoint)# exit
(config)# crypto isakmp policy 1
(config-isakmp)# authentication rsa-sig
(config-isakmp)# exit
			\end{lstlisting}
		\end{minipage}
	\end{center}
	\caption{Configuration de l'authentification via certificats}
	\label{configuration_authentification_certificats}
\end{figure}

\subsubsection{Mise en place côté client}


