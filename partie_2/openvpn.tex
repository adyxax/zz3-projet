\subsection{OpenVPN}

\subsubsection{Généralités}
OpenVPN est un outil Open-Source permettant de créer des tunnels sécurisés (SSL/TLS) à travers un réseau non sûr comme Internet. OpenVPN est à la fois facile à installer et à configurer, en plus d'être disponible sur à peut prêt toutes les plates-formes (Linux, Windows, BSD, Mac OS, Solaris). Le principe de configuration reste le même quelque soit la plate-forme utilisée.

OpenVPN est basé sur une architecture client-server. Le VPN fonctionne soit site-à-site, soit avec des clefs partagées. Les données sont tunnelisées sur un seul port, TCP ou UDP.

\subsubsection{Les deux types de VPN}
OpenVPN propose deux types de VPN : les VPN bridgés et les VPN routés. Dans le cas du mode bridgé le réseau virtuel créé devient une réelle extension du réseau local. L'avantage de ce mode est la facilité d'intégration de la solution VPN au sein de l'infrastructure déjà en place. Ce mode est d'ailleurs la seule option si pour une raison ou pour une autre des paquets broadcasts doivent traverser le VPN. Le principal inconvénient de ce mode apparait lors du passage à l'échelle : comme c'est le réseau local qui doit absorber les clients du VPN, il faut suffisament de ressources disponibles (adresses IP, etc.).

TODO : c'est pas tout à fait vrai

Le mode routé est le mode le plus utilisé. Bien que sa mise en place soit plus complexe, ce mode permet de faire du réseau virtuel un réseau à part du réseau local. On est ainsi capable de mettre en place une politique d'accès différente pour les utilisateurs connectés depuis l'extérieur, ce qui renforce encore la sécurité de l'infrastructure. Le second grand aventage des VPN routés est que le passage a l'échelle s'effectue ne pose aucun problème étant donné que l'on n'impacte pas l'utilisation du réseau local.

~

La figure \ref{tableau_types_vpn} résume les avantages et inconvénient de chacun des deux types de VPN :
\begin{figure}[H]
	\begin{center}
		\begin{tabular}{c|c}
			VPN bridgé & VPN routé \\
			\hline
			extension du réseau local & réseau à part \\
			mauvais passage à l'échelle & passage à l'échelle \\
			laisse passer les broadcasts & broadcasts impossibles \\
		\end{tabular}
	\end{center}
	\caption{Avantages et inconvénients des deux types de VPN}
	\label{tableau_types_vpn}
\end{figure}

Nous avons choisi de mettre en place une configuration de VPN routé, car mieux adaptée à l'utilisation que l'ISIMA pourrait en faire.

\subsubsection{Mise en place côté serveur}

La mise en place de l'infrastructure s'effectue en plusieures étapes. Tout d'abord nous installerons et configurerons OpenVPN sur le serveur, ensuite nous génèrerons les clefs et certificats de sécurité nécessaires, et enfin nous mettrons en place les outils nécessaires pour authentifier les utilisateurs via l'annuaire NIS de l'ISIMA.

Les interfaces de la machine sont configurées comme indiqué sur le schéma ref{TODO}, et résumé dans le tableau figure \ref{linux_interfaces} :

\begin{figure}[H]
	\begin{center}
		\begin{tabular}{c|c|l}
			Interface & Adresse IP & Commentaire \\
			\hline
			eth0 & 192.168.0.10 & Réseau interne étudiants \\
			eth1 & 192.168.102.121 & Réseau externe \\
			eth2 & 10.0.0.10 & Réseau interne profs \\
		\end{tabular}
	\end{center}
	\caption{Configuration des interfaces de la machine Linux}
	\label{linux_interfaces}
\end{figure}


\subsubsection{Installation et configuration d'OpenVPN}

\paragraph{Résolution des dépendances}
~

La machine fonctionne sous \texttt{Linux CentOS 5.1}. OpenVPN n'étant pas disponible directement dans les paquets de cette distribution (datant d'il y a presque deux ans à l'écriture de ces lignes), nous allons construire notre propre rpm. Les paquets requis pour mener à bien cette étape sont à installer grâce à la commande suivante :

\verb|[root@centosvpn ~]# yum install openssl-devel pam-devel rpm-build gcc-c++|

~

La version d'OpenVPN utilisée pour le projet est la 2.0.9 disponible sur \verb|http://openvpn.net/|. Celle-ci dépend des paquets \verb|lzo-devel-1.08-fr2.i386| et \verb|lzo-1.08-fr2.i386|, disponibles par exemple sur \verb|http://rpmfind.net/|.

\verb|[root@centosvpn ~]# wget ftp://rpmfind.net/linux/freshrpms/redhat/9/|

\verb|[root@centosvpn ~]# lzo/lzo-devel-1.08-fr2.i386.rpm|

\verb|[root@centosvpn ~]# rpm -ivh lzo-1.08-fr2.i386 lzo-devel-1.08-fr2.i386.rpm|

\paragraph{Compiler OpenVPN}
~

Lors de la configuration de notre maquette la version courante d'OpenVPN était la 2.0.9; adaptez les numéros de version avec celui de la dernière release stable d'OpenVPN. Les commandes suivantes permettent à la fois de la récupérer, de la compiler, et de l'installer :

\verb|[root@centosvpn ~]# wget http://openvpn.net/release/openvpn-2.0.9.tar.gz|

\verb|[root@centosvpn ~]# rpmbuild -tb openvpn-2.0.9.tar.gz|

\verb|[root@centosvpn ~]# rpm -ivh /usr/src/redhat/RPMS/i386/openvpn-2.0.9-1.i386.rpm|

\paragraph{Configuration de base}
~

Une instance d'OpenVPN ne peut gérer qu'un seul pool d'adresses à la fois, et donc un seul type de clients pour le VPN. La solution pour gérer à la fois les professeurs et les étudiants grâce à une même machine est donc de lancer deux instances du serveur en écoute sur un port différent. Nous allons donc utiliser deux fichiers de configuration distincts dont la figure \ref{configuration_base_openvpn} présente les paramètres qui leurs sont communs : Interface d'écoute, protocole de transport, etc.

\begin{figure}[H]
	\begin{center}
		\begin{minipage}{0.90\textwidth}
			\begin{lstlisting}[frame=trBL]
local 192.168.102.121
proto udp
dev tap
client-to-client
duplicate-cn
keepalive 10 120
comp-lzo
user nobody
group nobody
persist-key
persist-tun
status openvpn-status.log
verb 3
			\end{lstlisting}
		\end{minipage}
	\end{center}
	\caption{Configuration de base d'OpenVPN}
	\label{configuration_base_openvpn}
\end{figure}

~

Pour configurer correctement deux instances qui puissent cohabiter, celles-ci doivent se mettre en écoute sur un port différent. Nous avons choisi le port 1194 (port par défaut d'OpenVPN) pour le serveur profs, ainsi que le port 1195 pour le serveur étudiant. Les figures \ref{configuration_base_prof} et \ref{configuration_base_student} détaillent également la configuration des pool d'adresses à affecter aux clients, ainsi que les informations de routage à leur transmettre :

\begin{figure}[H]
	\begin{lstlisting}[frame=trBL]
port 1194
server 10.0.1.0 255.255.255.0
ifconfig-pool-persist ipp-profs.txt
push "route 10.0.0.0 255.255.255.0"
	\end{lstlisting}
	\caption{Configuration spécifique à \texttt{/etc/openvpn/server-prof.conf}}
	\label{configuration_base_prof}
\end{figure}
\begin{figure}[H]
	\begin{lstlisting}[frame=trBL]
port 1195
server 192.168.1.0 255.255.255.0
ifconfig-pool-persist ipp-profs.txt
push "route 192.168.0.0 255.255.255.0"
	\end{lstlisting}
	\caption{Configuration spécifique à \texttt{/etc/openvpn/server-student.conf}}
	\label{configuration_base_student}
\end{figure}



% ca /etc/openvpn/ca.crt
% cert /etc/openvpn/openvpn.crt
% key /etc/openvpn/openvpn.key  # This file should be kept secret
% dh /etc/openvpn/dh1024.pem
% tls-auth /etc/openvpn/ta.key 0 # This file is secret
% plugin /usr/share/openvpn/plugin/lib/openvpn-auth-pam.so login

\subsubsection{Génération des clefs et certificats de sécurité}

Nous allons maintenant utiliser openssl pour générer nos clefs et certificats de sécurité.

\subsubsection{Authentification via l'annuaire de l'ISIMA}

OpenVPN étant capable de réaliser une authentification PAM (méthode standart sur les systèmes UNIX), c'est la solution que nous avons retenue. Nous allons donc configurer un client NIS sur la machine de test, et indiquer à OpenVPN comment l'utiliser.

La première étape consiste à installer le client NIS si ce n'est pas déjà fait (paquet \texttt|ypserv|) et à faire entrer la machine dans le domaine NIS de l'ISIMA. \texttt{glouglou.isima.fr}.

\subsubsection{Configuration de démarrage de la machine}


\pagebreak


% \begin{figure}[H]
% 	\begin{center}
% 		\begin{minipage}{0.90\textwidth}
% 			\begin{lstlisting}[frame=trBL]
% 
% 			\end{lstlisting}
% 		\end{minipage}
% 	\end{center}
% 	\caption{Récupération des éléments de réponse dans un message de commande}
% 	\label{format_reponse_commande}
% \end{figure}
