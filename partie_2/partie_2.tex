\section{Mise en place des maquettes}

	Dans cette partie, nous allons détailler les trois architectures mise en place pour la création d'un VPN.Nous commenceront par détailler la solution VPN que Microsoft propose.

\subsection{Solution Windows}

	Pour ce projet, nous avons décider d'installer Windows Server 2003 Entreprise Edition, afin d'avoir un sytème d'exploitation récent. Toutes les fonctionnalités utilisées pour la création d'un VPN sont incluses dans cette version. Il n'y a pas besoin d'installer une application tierce.
	Voyons à présent les différents services nécessaires pour la mise en place du VPN de Microsoft.

\subsubsection{Services installés}

	Afin de pouvoir se mettre dans les conditions du réseau interne de l'ISIMA, nous avons décidé de configurer plusieurs services:
~


\begin{itemize}
	\item  - DHCP,
	\item  - DNS, 
	\item  - Active Directory,
	\item  - Routage et accès distant.(VPN)
\end{itemize}
~	

	Ces nombreux services sont nécessaires pour la création d'un VPN. A présent, détaillons la configuration de chacun de ses servives.

~

	Le service DHCP

~

Ce service est nécessaire pour l'attribution des adresses IP qui seront allouées aux différents clients lors de leurs connexion au serveur VPN.

\subsubsection{Solution dans son fonctionnement}
\subsubsection{Limites de la solution}

\subsection{Solution Linux}
\subsubsection{Configuration du VPN}
\subsubsection{Sécurité+protocoles}
\subsubsection{Limites de la solution}

\section{OpenVPN}

\subsection{Introduction}

\subsubsection{Généralités}
OpenVPN est un outil Open-Source permettant de créer des tunnels sécurisés (SSL/TLS) à travers un réseau non sûr comme Internet. OpenVPN est à la fois facile à installer et à configurer, en plus d'être disponible sur à peut prêt toutes les plates-formes (Linux, Windows, BSD, Mac OS, Solaris). Le principe de configuration reste le même quelque soit la plate-forme utilisée.

OpenVPN est basé sur une architecture client-server. Le VPN fonctionne soit site-à-site, soit avec des clefs partagées. Les données sont tunnelisées sur un seul port, TCP ou UDP.

\subsubsection{Les deux types de VPN}
Il existe deux types de VPN : les VPN bridgés et les VPN routés. Dans le cas du mode bridgé le réseau virtuel créé devient une réelle extension du réseau local. L'avantage de ce mode est la facilité d'intégration de la solution VPN au sein de l'infrastructure déjà en place. Ce mode est d'ailleurs la seule option si pour une raison ou pour une autre des paquets broadcasts doivent traverser le VPN. Le principal inconvénient de ce mode apparait lors du passage à l'échelle : comme c'est le réseau local qui doit absorber les clients du VPN, il faut suffisament de ressources disponibles (adresses IP, etc.).

Le mode routé est le mode le plus utilisé. Bien que sa mise en place soit plus complexe, ce mode permet de faire du réseau virtuel un réseau à part du réseau local. On est ainsi capable de mettre en place une politique d'accès différente pour les utilisateurs connectés depuis l'extérieur, ce qui renforce encore la sécurité de l'infrastructure. Le second grand aventage des VPN routés est que le passage a l'échelle s'effectue ne pose aucun problème étant donné que l'on n'impacte pas l'utilisation du réseau local.

~

La figure \ref{tableau_types_vpn} résume les avantages et inconvénient de chacun des deux types de VPN :
\begin{figure}[H]
	\begin{center}
		\begin{tabular}{c|c}
			VPN bridgé & VPN routé \\
			\hline
			extension du réseau local & réseau à part \\
			mauvais passage à l'échelle & passage à l'échelle \\
			laisse passer les broadcasts & broadcasts impossibles \\
		\end{tabular}
	\end{center}
	\caption{Avantages et inconvénients des deux types de VPN}
	\label{tableau_types_vpn}
\end{figure}

Nous avons choisi de mettre en place une configuration de VPN routé, car mieux adaptée à l'utilisation que l'ISIMA pourrait en faire.

\subsection{Mise en place côté serveur}

La mise en place de l'infrastructure s'effectue en plusieures étapes. Tout d'abord nous installerons et configurerons OpenVPN sur le serveur, ensuite nous génèrerons les clefs et certificats de sécurité nécessaires, et enfin nous mettrons en place les outils nécessaires pour authentifier les utilisateurs via l'annuaire NIS de l'ISIMA.

\subsubsection{Installation et configuration d'OpenVPN}

La machine fonctionne sous \texttt{Linux CentOS 5.1}. OpenVPN n'étant pas disponible directement dans les paquets de cette distribution, nous allons construire notre propre rpm. Les paquets requis pour mener à bien cette étape sont à installer grâce à la commande suivante :

\verb|[root@centosvpn ~]# yum install openssl-devel pam-devel rpm-build gcc-c++|

~

La version d'OpenVPN utilisée pour le projet est la 2.0.9 disponible sur \verb|http://openvpn.net/|. Celle-ci dépend des paquets \verb|lzo-devel-1.08-fr2.i386| et \verb|lzo-1.08-fr2.i386|, disponibles par exemple sur \verb|http://rpmfind.net/|.

\verb|[root@centosvpn ~]# wget ftp://rpmfind.net/linux/freshrpms/redhat/9/lzo/lzo-devel-1.08-fr2.i386.rpm|

\subsubsection{Génération des clefs et certificats de sécurité}

\subsubsection{Authentification via l'annuaire de l'ISIMA}

OpenVPN étant capable de réaliser une authentification PAM (méthode standart sur les systèmes UNIX), c'est la solution que nous avons retenue. Nous allons donc configurer un client NIS sur la machine de test, et indiquer à OpenVPN comment l'utiliser.

La première étape consiste à installer l'annuaire NIS si ce n'est pas déjà fait (paquet \texttt|ypserv|) à faire entrer la machine dans le domaine NIS de l'ISIMA : .




\pagebreak


% \begin{figure}[H]
% 	\begin{center}
% 		\begin{minipage}{0.90\textwidth}
% 			\begin{lstlisting}[frame=trBL]
% 
% 			\end{lstlisting}
% 		\end{minipage}
% 	\end{center}
% 	\caption{Récupération des éléments de réponse dans un message de commande}
% 	\label{format_reponse_commande}
% \end{figure}


\subsection{Solution Cisco}
\subsubsection{Protocoles utilisés}
\subsubsection{Système d'authentification}
\subsubsection{Configuration du routeur}
\subsubsection{Limites de la solution}

\subsection{Solution Cisco}

\subsubsection{Généralités}
ipsec esp-sha1

\subsubsection{Mise en place côté serveur}


\paragraph{Configuration de base}
~

Commençons par configurer les interfaces du routeur. L'interface connectée au réseau de l'ISIMA récupère son adresse IP via DHCP. Cela :
\begin{figure}[H]
	\begin{center}
		\begin{minipage}{0.90\textwidth}
			\begin{lstlisting}[frame=trBL]
(config)# hostname CISCOVPN
(config)# enable secret cisco
(config)# no ip domain-lookup
(config)# interface FastEthernet0/0
(config-if)# description interface to the external network
(config-if)# ip address dhcp
(config-if)# no shutdown
(config-if)# exit
(config)# interface FastEthernet0/1.1
(config-if)# description interface to the prof network
(config-if)# encapsulation dot1Q 333
(config-if)# ip address 10.0.0.1 255.255.255.0
(config-if)# exit
(config)# interface FastEthernet0/1.2
(config-if)# description interface to the student network
(config-if)# encapsulation dot1Q 111
(config-if)# ip address 192.168.0.1 255.255.255.0
(config-if)# exit
(config)# interface FastEthernet0/1
(config-if)# no shutdown
(config-if)# exit
			\end{lstlisting}
		\end{minipage}
	\end{center}
	\caption{Configuration des interfaces}
	\label{configuration_interfaces}
\end{figure}

~

Configurons maintenant les accès à distance au routeur :
\begin{figure}[H]
	\begin{center}
		\begin{minipage}{0.95\textwidth}
			\begin{lstlisting}[frame=trBL]
(config)# banner login #Unauthorized access prohibited - F5 only!#
(config)# banner motd #
This router is part of a wonderfull ZZ3F5 project for 2008-2009.
If you have any question, comment, insults, whatsoever...
please contact coscia@poste.isima.fr and dessaux@poste.isima.fr.
Thank you if you read this till the end.#
(config)# enable secret cisco
(config)# line con 0
(config-line)# logging synchronous
(config-line)# password cisco
(config-line)# login
(config-line)# exit
(config)# line vty 0 4
(config-line)# transport input telnet
(config-line)# password cisco
(config-line)# login
(config-line)# exit
(config)# service password-encryption
			\end{lstlisting}
		\end{minipage}
	\end{center}
	\caption{Configuration de l'accès à distance}
	\label{configuration_acces_a_distance}
\end{figure}

\paragraph{Configuration de l'authentification Radius}
~

Tout d'abord le serveur FreeRadius.
Authentication PAP entre radius server and cisco router :
Ajouter \verb|ciscovpn User-Password := "isima"|  au début du fichier \verb|/etc/raddb/users|, où ciscovpn est le hostname du routeur et isima le mot de passe qui lui sera associé.
Ajouter quelques lignes pour autoriser le router à interroger la base RADIUS dans le fichier /etc/raddb/clients.conf, et commenter la ligne shadow dans radiusd.conf

~


\begin{figure}[H]
	\begin{center}
		\begin{minipage}{0.90\textwidth}
			\begin{lstlisting}[frame=trBL]
(config)# aaa new-model
(config)# radius-server host 192.168.102.121 auth-port 1812
acct-port 1813 key isima
(config)# ip radius source-interface FastEthernet 0/0
(config)# aaa group server radius RadiusServer
(config-sg-radius)# radius-server host 192.168.102.121 auth-port
1812 acct-port 1813
(config-sg-radius)# exit
(config)# aaa authentication login default group RadiusServer
			\end{lstlisting}
		\end{minipage}
	\end{center}
	\caption{Configuration de l'authentification Radius}
	\label{configuration_authentification_radius}
\end{figure}


\paragraph{Configuration d'IPsec}
~

Les lignes qui suivent permettent de configurer la cryptographie isakmp :
\begin{itemize}
	\item algorithme de chiffrement triple DES.
	\item algorithme de hashage sha-1.
	\item authentification via clefs partagées.
	\item Diffie-Hellman 1024 bits.
	\item durée de vie du contexte cryptographique égale à une journée.
	\item utilisation du hostname plutôt que de l'adresse IP pour protéger les échanges.
\end{itemize}


\begin{figure}[H]
	\begin{center}
		\begin{minipage}{0.90\textwidth}
			\begin{lstlisting}[frame=trBL]
(config)# crypto isakmp policy 1
(config-isakmp)# encryption 3des
(config-isakmp)# hash sha
(config-isakmp)# authentication pre-share
(config-isakmp)# group 2
(config-isakmp)# lifetime 86400
(config-isakmp)# exit
(config)# crypto isakmp identity hostname
			\end{lstlisting}
		\end{minipage}
	\end{center}
	\caption{Configuration IKE}
	\label{configuration_ike}
\end{figure}

Ajoutons les pools DHCP qui fourniront leurs adresses IP aux étudiants et aux professeurs :
\begin{figure}[H]
	\begin{center}
		\begin{minipage}{0.90\textwidth}
			\begin{lstlisting}[frame=trBL]
(config)# ip local pool profs 10.0.1.20 10.0.1.254
(config)# crypto isakmp client configuration group profs
(config-isakmp-group)# key isimaprofs
(config-isakmp-group)# dns 10.0.0.11
(config-isakmp-group)# domain isima.fr
(config-isakmp-group)# pool profs
(config-isakmp-group)# exit

(config)# ip local pool students 192.168.1.20 192.168.1.254
(config)# crypto isakmp client configuration group students
(config-isakmp-group)# key isimastudents
(config-isakmp-group)# dns 192.168.1.11
(config-isakmp-group)# domain isima.fr
(config-isakmp-group)# pool students
(config-isakmp-group)# exit
			\end{lstlisting}
		\end{minipage}
	\end{center}
	\caption{Configuration des pools utilisateurs}
	\label{configuration_pools_utilisateurs}
\end{figure}

% TODO certificats

Configurons maintenant la police IPsec :
\begin{itemize}
	\item Mise en place de l'ACL pour indiquer que l'on veut filtrer l'ensemble du trafic IP.
	\item Encapsulation ESP, chiffrement 3des, intégrité sha-1.
	\item Mode tunnel (ie au niveau 3, par opposition au mode transport au niveau 4).
	\item Configuration de l'authentification des profs et des étudiants via le serveur radius.
\end{itemize}

\begin{figure}[H]
	\begin{center}
		\begin{minipage}{1\textwidth}
			\begin{lstlisting}[frame=trBL]
(config)# access-list 101 permit ip any any
(config)# crypto ipsec transform-set policy esp-3des esp-sha-hmac
(cfg-crypto-trans)# mode tunnel
(cfg-crypto-trans)# exit
(config)# crypto dynamic-map prof-map 1
(config-crypto-map)# set transform-set policy
(config-crypto-map)# exit

(config)# crypto map prof-map
(config)# crypto map prof-map 1 ipsec-isakmp dynamic prof-map
(config)# crypto map prof-map client authentication list RadiusServer
(config)# cryto map prof-map client configuration address respond
(config)# crypto map prof-map isakmp authorization list 101

(config)# aaa authorization network 101 local
			\end{lstlisting}
		\end{minipage}
	\end{center}
	\caption{Configuration de la police IPsec}
	\label{configuration_police_ipsec}
\end{figure}

une seule crypto map, identification en fonction du group dans le client (appel direct au pool car la crypto map n'y est pas liée, ça marche). Le problème c'est qu'un étudiant peut facilement se faire passer pour un prof.

\begin{figure}[H]
	\begin{center}
		\begin{minipage}{1\textwidth}
			\begin{lstlisting}[frame=trBL]
(config)# crypto ca identity isima.fr
(ca-trustpoint)# enrollment url
http://192.168.102.250/certsrv/mscep/mscep.dll
(ca-trustpoint)# exit
(config)# crypto ca authenticate isima.fr
(config)# crypto ca trustpoint isima.fr
(ca-trustpoint)# crl optional
(ca-trustpoint)# exit
(config)# crypto isakmp policy 1
(config-isakmp)# authentication rsa-sig
(config-isakmp)# exit
			\end{lstlisting}
		\end{minipage}
	\end{center}
	\caption{Configuration de l'authentification via certificats}
	\label{configuration_authentification_certificats}
\end{figure}

\subsubsection{Mise en place côté client}




\pagebreak
