\section{Mise en place des maquettes}

	Dans cette partie, nous allons détailler les trois architectures mise en place pour la création d'un VPN.Nous commenceront par détailler la solution VPN que Microsoft propose.

\subsection{Solution Windows}

	Pour ce projet, nous avons décider d'installer Windows Server 2003 Entreprise Edition, afin d'avoir un sytème d'exploitation récent. Toutes les fonctionnalités utilisées pour la création d'un VPN sont incluses dans cette version. Il n'y a pas besoin d'installer une application tierce.
	Voyons à présent les différents services nécessaires pour la mise en place du VPN de Microsoft.

\subsubsection{Services installés}

	Afin de pouvoir se mettre dans les conditions du réseau interne de l'ISIMA, nous avons décidé de configurer plusieurs services:
~


\begin{itemize}
	\item  - DHCP,
	\item  - DNS, 
	\item  - Active Directory,
	\item  - Routage et accès distant.(VPN)
\end{itemize}
~	

	Ces nombreux services sont nécessaires pour la création d'un VPN. A présent, détaillons la configuration de chacun de ses servives.

~

	Le service DHCP

~

Ce service est nécessaire pour l'attribution des adresses IP qui seront allouées aux différents clients lors de leurs connexion au serveur VPN.

\subsubsection{Solution dans son fonctionnement}
\subsubsection{Limites de la solution}

\subsection{Solution Linux}
\subsubsection{Configuration du VPN}
\subsubsection{Sécurité+protocoles}
\subsubsection{Limites de la solution}

\subsection{Solution Cisco}
\subsubsection{Protocoles utilisés}
\subsubsection{Système d'authentification}
\subsubsection{Configuration du routeur}
\subsubsection{Limites de la solution}

\pagebreak
