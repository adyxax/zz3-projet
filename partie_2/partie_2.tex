\section{Mise en place des maquettes}

	Dans cette partie, nous allons détailler les trois architectures mise en place pour la création d'un VPN.Nous commenceront par détailler la solution VPN que Microsoft propose.

\subsection{Solution Windows}

	Pour ce projet, nous avons décider d'installer Windows Server 2003, afin d'avoir un serveur ``léger''. Toutes les fonctionnalités utilisées pour la création d'un VPN sont incluses dans cette version. Il n'y a pas besoin d'installer une application tierce.
	Voyons à présent les différents services nécessaires pour la mise en place du VPN de Microsoft.

\subsubsection{Services installés}

	
\subsubsection{Solution dans son fonctionnement}
\subsubsection{Limites de la solution}

\subsection{Solution Linux}
\subsubsection{Configuration du VPN}
\subsubsection{Sécurité+protocoles}
\subsubsection{Limites de la solution}

\subsection{Solution Cisco}
\subsubsection{Protocoles utilisés}
\subsubsection{Système d'authentification}
\subsubsection{Limites de la solution}

\pagebreak
