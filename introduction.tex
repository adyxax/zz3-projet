\section*{Introduction}
\addcontentsline{toc}{section}{Introduction}

Dans le cadre de notre 3ème année, notre cursus prévoit la réalisation d’un projet d’une durée de 120 heures répartie du mois d’octobre 2008 à mars 2009. 

Notre sujet porte sur l'étude de la mise en place d'une solution VPN pour l'école. Ceci permettra aux étudiants et aux professeurs de se connecter sur le réseau de l'ISIMA de manière sécurisée.


Dans un premier temps, nous évoquerons les différents types de VPN ainsi que sur les protocoles utilisées. Nous détaillerons notre maquette d'un point de physique et logique grâce à deux schémas explicatifs.


Dans un second temps, nous détaillerons la configuration de chaque solution d'un point de vue serveur et client. Pour chaque solution nous ferons un bilan et nous en évoquerons les limites. 


Enfin, les différentes solutions seront confrontées selon plusieurs critères prenant en compte les contraintes de sécurité des VPN. Elles seront d'abord évaluées en termes de complexité d'installation et d'administration, puis en termes de performances, de niveau de sécurité fourni et de facilité d'utilisation. Ces différents résultats nous permetterons de choisir la meilleure solution.




\pagebreak
