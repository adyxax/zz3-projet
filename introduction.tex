\section*{Introduction}
\addcontentsline{toc}{section}{Introduction}

~

Les VPN se sont développés parallèlement à l'essor des technologies de communication. L'informatique s'est vite révélé être un formidable outil, mais le besoin d'échanger des informations de façon sécurisée a vite tempéré les esprits. De nombreuses solutions soit-disant sécurisées se sont rapidement développées, mais beaucoup ont échoué car elles considéraient la sécurité comme une fin et non comme un moyen.

~

Les technologies d'accès distant comme les VPN ont été la réponse à une approche réfléchie de la problématique de la sécurité, ce qui explique que leur essor se poursuit encore à l'heure actuelle. Cette problématique se pose également pour l'ISIMA, et c'est ce qui a motivé notre projet. Un tel accès permettrait d'améliorer concrètement la façon dont étudiants et professeurs travaillent, en leur fournissant un moyen adapté, flexible et sûr d'accéder au réseau de l'école depuis l'extérieur.

~

Dans un premier temps, nous présenterons plus en détail ce qu'est un VPN, nous évoquerons les différentes solutions sur le marché ainsi que sur les protocoles sur lesquels elles reposent. Nous présenterons également de façon schématique la plate-forme de test que nous utiliserons.

Dans un second temps, nous entrerons dans les détails de la configuration de chaque solution, tant d'un point de vue serveur que d'un point de vue client. Pour chacune d'entre-elles nous effectuerons un bilan et en dégagerons les limites.

Finalement, les différentes solutions seront confrontées selon plusieurs critères prenant en compte les contraintes de sécurité des VPN. Elles seront d'abord évaluées en termes de complexité d'installation et d'administration, puis en termes de performances, de niveau de sécurité fourni et de facilité d'utilisation. Ces différents résultats nous permettront de déterminer le meilleur choix d'implémentation pour l'ISIMA.

\pagebreak
